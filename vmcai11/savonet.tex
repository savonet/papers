\documentclass{llncs}

\newcommand{\liquidsoap}{Liquidsoap}
\newcommand{\savonet}{Savonet}
\newcommand{\eg}{e.g.~}
\newcommand{\cf}{cf.~}

\usepackage{url}

\title{Savonet}
\author{David Baelde \and Romain Beauxis \and Samuel Mimram}

\begin{document}
\maketitle

\section*{Introduction}

The widespread of broadband Internet access and digital media during the last
decade has attracted a lot of attention on their potential applications.
Classical applications from the analog era, such as television, 
radio broadcasting and phone communications are being ported and enhanced to the digital
world.

Additionally, while analog applications where mostly based on hardware implementations,
their digital counterpart are often software implementations, \eg{} hardware PBX for phone
communications being replaced by software like Asterisk and IP phones. These software
implementation usually offer much more flexibility and allow updates, both 
for bugfix and new features, at virtually no costs.

In this context of improvement, update and enhancement of old technologies, we 
are interested in audio and video broadcasting. Creating and broadcasting 
a stream of multimedia data with recent computers has become very easy.
Yet, the software technologies available to perform such tasks
have brought very few new ideas related to the new possibilities offered
by modern computing technologies.

Designing a multimedia stream can require a lot of flexibility 
in order to achieve the program that one may want to create. For instance,
a radio stream may have jingles announcing next coming shows
or commercials. It may also play those jingles at a regular 
interval of time, between songs or on top of them. Also, a radio program may be 
composed of automatic playlist for a certain period, \eg{} during the nigh, 
and live shows during the day.

Similarly, one may want to control and process the data before broadcasting
it to the public, performing tasks like:
\begin{itemize}
 \item Audio volume normalization.
 \item Cross-fading between tracks, possibly parameterized by the respective volumes
of the old and new tracks or predefined settings.
 \item Cut blanks, either in automatic file streams or during live shows.
\end{itemize}

Those examples, among many others, express the need for flexible
and adaptative solutions for creating and broadcasting multimedia data.
Classical tools to broadcast multimedia data over the Internet, like
Darkice, Ezstream, VideoLAN, Rivendell, SAM Broadcaster, \dots mostly consist 
of straight-forward adaptation of classical streaming technologies, whose 
paradigms are based on predefined interfaces, such as a virtual mixing console
or static file-based setups. Those tools, although quite powerful, are usually very hard 
to adapt to a particular need.

However, a particularly important aspect of modern software technologies is programming
languages. Whether general or applicative, programming languages are the first 
class tools to release creativity and flexibility for creating new software applications.
Programming languages have been used in various practical contexts to bring flexibility
and overcome static pre-defined paradigms. One may, for instance, think of the Perl 
language, invented to allow powerful and flexible word-based treatments, or the PHP
language, invented to create easily dynamic webpages.

The application we present in this paper brings to the domain of multimedia
broadcasting the ideas and technologies of software engineering. It originated 
after realizing that the existing tools for digital broadcasting where not flexible
and expressive enough to fit the authors' need. In order to overcome those shortcomings,
an applied language was developed which, as for Perl and PHP, includes first-class
notions of streams, with operators to create them, combine them and modify them.
Using this language, the possibilities for designing and creating a multimedia stream
are very broad and allow creative innovations.

Another important aspect of this application is its potential users. Indeed,
multimedia stream designer are not often programming language specialist. More
over, creating an Internet radio should not require advanced programming skills.
For there reasons, although the language is intended to be powerful, it should 
also be relatively easy to use, at least for building a basic stream.

The language, called \liquidsoap{} \cite{liquidsoap}, is a functional language, implemented as 
a script language, with an interpreter written in OCaml. In the following,
we present this language, show some of its applications and explain the 
related theoretical interesting considerations.
TODO: expand this part when we know what we are talking about :-)

\section{The \liquidsoap{} language}
\label{sec:lang}

The \liquidsoap{} language was developed in the \savonet{} project \cite{savonet}
which, beside the \liquidsoap{} language, also provides multimedia-related
applications, such as bindings for several multimedia libraries in OCaml or
GUI interfaces for managing multimedia stream. The whole project contains
more than 50~000 lines of code (LoC), among them 40~000 are written in OCaml
and 8~000 in C (TODO: update). The \savonet{} project is still very active
and has received many contributions and feedbacks from users, including
large-scale applications (TODO: refs).

Using the \liquidsoap{} language, one can very quickly create an initial
simple stream. For instance, the following code\footnote{The examples is this paper
are given using the syntax of the (currently) developpement version of the language. 
At the time this text was written, the latest released version was 0.9.2 and 
the current developpement version is intended to be released as 1.0.}, when executed,
creates a stream containing a playlist of files located in a given directory and sends 
it to an Icecast server:
\begin{verbatim}
l = playlist("/path/to/files")
output.icecast(%vorbis,host="www.radio.com", name="ma_radio", l)
\end{verbatim}


\bibliographystyle{abbrv}
\bibliography{biblio}
\end{document}
