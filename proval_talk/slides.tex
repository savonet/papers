\documentclass{beamer}

\usepackage[utf8]{inputenc}
\usepackage[french]{babel}

\usetheme{Pittsburgh}
\useinnertheme{default}
\setbeamertemplate{footline}[page number]
\setbeamertemplate{navigation symbols}{}

%\usepackage[utf8]{inputenc}
\usepackage{listings}
\lstset{ %
language=Ruby,                % choose the language of the code
basicstyle=\footnotesize,       % the size of the fonts that are used for the code
numbers=left,                   % where to put the line-numbers
numberstyle=\footnotesize,      % the size of the fonts that are used for the line-numbers
stepnumber=0,                   % the step between two line-numbers. If it's 1 each line 
                                % will be numbered
numbersep=5pt,                  % how far the line-numbers are from the code
backgroundcolor=\color{white},  % choose the background color. You must add \usepackage{color}
showspaces=false,               % show spaces adding particular underscores
showstringspaces=false,         % underline spaces within strings
showtabs=false,                 % show tabs within strings adding particular underscores
frame=none,	                % adds a frame around the code
tabsize=2,	                % sets default tabsize to 2 spaces
captionpos=b,                   % sets the caption-position to bottom
breaklines=true,                % sets automatic line breaking
breakatwhitespace=false,        % sets if automatic breaks should only happen at whitespace
title=\lstname,                 % show the filename of files included with \lstinputlisting;
                                % also try caption instead of title
escapeinside={\%*}{*)},         % if you want to add a comment within your code
emph={add,output,input,icecast,single,playlist,file,
              fallback,crossfade,http,fade,initial,final,cross},            % if you want to add more keywords to the set
keywordstyle=\color[rgb]{0,0.5,0.3},
emphstyle=\color{red},
stringstyle=\color{blue},
}

\newcommand{\kw}[1]{{\color{red} #1}}

\usepackage{tikz}
\usepackage[all]{xy}

\renewcommand{\emph}[1]{\alert{#1}}
\renewcommand{\textbf}[1]{{\color{blue} #1}}

\usepackage{mathpartir}

\newcommand{\liquidsoap}{Liquidsoap}
\newcommand{\savonet}{Savonet}
\newcommand{\eg}{e.g.~}
\newcommand{\cf}{cf.~}
\newcommand{\letin}[3]{\text{\texttt{let }} #1 = #2 \text{\texttt{ in }} #3}
\newcommand{\univ}[2]{\forall #1.~ #2}
\newcommand{\regle}[1]{\text{\textrm{(#1)}}}
\newcommand{\mabs}[2]{{\color{red}\lambda\{}#1{\color{red}\}.}#2}
\newcommand{\tmabs}[2]{\{#1\}\to #2}
\newcommand{\mapp}[2]{#1{\color{red}\{}#2{\color{red}\}}}
\renewcommand{\vec}[1]{\overrightarrow{#1}}
\newcommand{\TODO}[1]{\textbf{TODO~: }{#1}}
\newcommand{\FV}[1]{\mathcal{FV}(#1)}
\newcommand{\eqdef}{:=}
\newcommand{\interp}[1]{\llbracket{#1}\rrbracket}

\newcommand{\sub}{\leq}

\title{\emph{\LARGE Liquidsoap} \\
  Un langage de programmation \\ pour les flux~multimedia}
\date{David Baelde} % pour le séminaire Proval du 20 janvier

\begin{document}

\begin{frame}
  \maketitle
\end{frame}

\begin{frame}{Introduction}

  \begin{block}{Radio web}
  \begin{itemize}
  \item \emph{Production}, diffusion et écoute de flux audio
  \item Outils à \emph{utilisation unique}, graphiques ou pas
    % légers et communicants (ices, ezstream, darkice)
    % ou graphiques plus complets (rivendell, winradio).
  \end{itemize}
  \end{block}

  \begin{block}{Liquidsoap}
  \begin{enumerate}
  \item Interfaces: bibliothèques, applications externes et distantes
  \item Modèle: Une notion abstraite de source (générateur de flux)
  \item \emph{Langage}: accès libre et sûr au modèle
  \end{enumerate}
  \end{block}

  \begin{block}{Historique}
  \begin{itemize}
  \item Depuis 2003 (ENS Lyon) avec S.~Mimram, R.~Beauxis
  \item 100k LOC dans trunk, 58k lignes de Caml pour liquidsoap
  \item Stable, ``nombreux'' utilisateurs
    % indépendantes, étudiantes, commerciales
    % musique, politique, art
  \end{itemize}
  \end{block}

\end{frame}


%% =======================================================================

\begin{frame}[fragile]{Principe}

\vspace{0.5cm}

\[
\xymatrix{
  *+[F]{\mathtt{input.http}} \ar[r] &
     *+[F]{\mathtt{fallback}} \ar[r]\ar[dr] &
     *+<7pt>[F=]{\mathtt{output.icecast}} \\
  *+[F]{\mathtt{playlist}}\ar[ur] &
     % *+[F]{\mathtt{crossfade}}\ar[u]
     &
     *+<7pt>[F=]{\mathtt{output.file}}\\
}
\]

\vspace{1cm}

\only<1>{\begin{block}{Une source, c'est quoi?}
  \begin{itemize}
    \item Flux d'échantillons, avec pistes et meta-données
    \item Peut être temporairement indisponible
    \item Passif (paresseux) ou actif (typiquement, sorties)
  \end{itemize}
\end{block}}

\pause

\begin{block}{Script}
Sources représentées par un \emph{type abstrait}
\begin{lstlisting}
live = input.http("http://server:8000/live.ogg")
music = playlist("/path/to/music")
s = fallback(track_sensitive=false,[live,music])
output.icecast(%vorbis,mount="radio.ogg",s)
output.file(%mp3(bitrate=16),"/path/to/backup.ogg",s)
\end{lstlisting}
\end{block}

\end{frame}

\begin{frame}[fragile]{Transitions}

  \begin{center}
  \begin{tikzpicture}[xscale=0.8,yscale=0.8]
  \draw[->] (0,0) -- (0,2.5);
  \draw (-0.1,2) -- (0.1,2);
  \draw (0,2) node[anchor=east]{100};
  \draw (0,0) node[anchor=east]{0};
  \draw[->] (0,0) -- (7.5,0);
  \foreach \x in {1,2,3,4,5,6,7} \draw (\x,-0.1) -- (\x,0.1);
  \draw (0,2.5) node[anchor=south]{volume (\%)};
  \draw (7.5,0) node[anchor=west]{time (sec)};
  \draw (0,2) -- (2,2) -- (5,0);
  \draw (2,0) -- (4,2) -- (7.5,2);
  \draw (1,2) node[anchor=south]{old};
  \draw (6,2) node[anchor=south]{new};
  \end{tikzpicture}
  \end{center}

\vfill

  \begin{lstlisting}
  def f(old,new) =
    add([fade.initial(duration=2.,new),
         fade.final(duration=3.,old)])
  end
  s = fallback(track_sensitive=false,
               transitions=[f,f],
               [live,music])
  \end{lstlisting}

\end{frame}

%% =======================================================================

\begin{frame}[fragile]{Le langage liquidsoap}

  \begin{block}{Pourquoi un nouveau langage}
  \begin{itemize}
    \item Fonctionnel, statiquement typé, étiquettes
\begin{center}\tiny\begin{verbatim}
$ liquidsoap -h cross
Generic cross operator, allowing the composition of the N last seconds of a track with
the beginning of the next track.
Type: (?id:string, ?duration:float, ?override:string, ?inhibit:float, ?minimum:float,
       ?conservative:bool, ?active:bool, ((source('a), source('a))->source('a)),
       source('a))->source('a)
Parameters:
* duration : float (default 5.)
    Duration in seconds of the crossed end of track. This value can be changed on a ...
* override : string (default "liq_start_next")
    Metadata field which, if present and containing a float, overrides the 'duration' ...
* inhibit : float (default -1.)
    Minimum delay between two transitions. It is useful in order to avoid that a ...
* minimum : float (default -1.)
    Minimum duration (in sec.) for a cross: If the track ends without any warning ...
* conservative : bool (default false)
    Do not trust remaining time estimations, always buffering data in advance. This ...
...
\end{verbatim}\end{center}
    \item Besoin d'analyses plus ou moins statiques
    \item Système de type spécifique
  \end{itemize}
  \end{block}

  % Pas de type principal, mais on peut plonger
  % En pratique: inférence limitée aux cas simples.

\end{frame}

\begin{frame}[fragile]{Les étiquettes d'OCaml}

\vspace{0.5cm}

  \begin{block}{Sucre syntaxique typé, éliminé à la compilation}
  \vspace{-0.3cm}\begin{verbatim}
# let app f = f ~a:1 ~b:2 ;;
val app : (a:int -> b:int -> 'a) -> 'a = <fun>
# app (fun ~b ~a -> a+b) ;;
This function should have type a:int -> b:int -> 'a.
  \end{verbatim}
  \end{block}

  \begin{block}{Multi-application cachée, arité non nulle}
  \vspace{-0.3cm}\begin{verbatim}
# let f ?(a=false) () = a ;;
val f : ?a:bool -> unit -> bool = <fun>
# f () ~a:true ;;
- : bool = true
# (f ()) ~a:true ;;
This expression is not a function, it cannot be applied.
  \end{verbatim}
  \end{block}

\end{frame}

\begin{frame}{Termes}

Petit ML avec multi-abstraction et application:
\begin{eqnarray*}
  M &::=& v \\
  &|& x \\
  &|& \letin{x}{M}{M} \\
  &|& \lambda \{\ldots, l_i:x_i, \ldots, l_j:x_j=M, \ldots\}.M \\
  &|& M\{l_1=M,\ldots,l_n=M\}
\end{eqnarray*}

\vfill

Modulo \emph{permutation d'étiquettes distinctes}:
\[ \small
 \begin{array}{rcl}
 \lambda \{ \Gamma, l:x, l':x', \Delta \}.M & \equiv &
 \lambda \{ \Gamma, l':x', l:x, \Delta \}.M \\
 \lambda \{ \Gamma, l:x=N, l':x', \Delta \}.M & \equiv &
 \lambda \{ \Gamma, l':x', l:x=N, \Delta \}.M \\
 \lambda \{ \Gamma, l:x=N, l':x'=N', \Delta \}.M & \equiv &
 \lambda \{ \Gamma, l':x'=N', l:x=N, \Delta \}.M
 \end{array}
\]

\end{frame}

\begin{frame}{Réduction}
On a trois règles de réduction:
\[
 \letin{x}{M}{N}
   {\color{red}\;\leadsto\;}
   N[M/x]
\]
L'application peut être \emph{partielle}
  si $\Gamma$ est \emph{ineffaçable}:
\[
 (\lambda\{\vec{l_i:x_i,l_j:x_j=M_j},\Gamma\}.M)\{\vec{l_i=N_i}\}
   {\color{red}\;\leadsto\;}
   \lambda\{\Gamma\}.(M[\vec{N_i/x_i}])
  \]
ou \emph{totale} sinon:
\[
(\lambda\{\vec{l_i:x_i,l_j:x_j=P_j},\vec{l'_k:y_k=M_k}\}.M)\{\vec{l_i=N_i}\}
   {\color{red}\;\leadsto\;}
   M[\vec{N_i/x_i},\vec{M_k/y_k}]
 \]

\begin{block}{Exemple}
\begin{itemize}
  \item $F := \mabs{l_1:x,l_2:y={\color{blue}12},l_3:z={\color{blue}13}}{M}$
  \item $\mapp{F}{l_3={\color{blue}3}} \leadsto \mabs{l_1:x,l_2:y={\color{blue}12}}{M[{\color{blue}3}/z]}$
  \item $\mapp{\mapp{F}{l_3={\color{blue}3}}}{l_1={\color{blue}1}} \leadsto M[{\color{blue}1}/x,{\color{blue}12}/y,{\color{blue}3}/z]$
\end{itemize}
\end{block}
\end{frame}

\begin{frame}{Typage}

\begin{block}{Types}
\begin{eqnarray*}
  t &::=& \iota
  \quad\mathop{|}\quad \alpha
  \quad\mathop{|}\quad
     \{\ldots, l_i:t_i,\ldots, ?l_j:t_j, \ldots\} \rightarrow t \\
 \sigma &::=& t
    \quad\mathop{|}\quad \univ{\alpha}{\sigma} \label{tt:univ}
\end{eqnarray*}
\end{block}

\vfill

\begin{block}{Règles}
\[
  \inferrule{%
    \Gamma\vdash N_i:t_i \mbox{ pour chaque $i$} \\ %
    \Gamma \vdash M:\{\ldots, l_i:t_i\ldots, ?l_j:t_j,\ldots\}\rightarrow t
  }{%
    \Gamma \vdash M \{ \ldots, l_i=N_i, \ldots, l_j=N_j, \ldots \} : t %
  }
\]
\vspace{0.1cm}
\[ \small
  \inferrule{ %
    \Gamma, \ldots, x_i:t_i, \ldots, x_j:t_j, \ldots \vdash M : t
    \\ \cdots \\
    \Gamma\vdash M_j:t_j
    \\ \cdots %
  }{%
    \Gamma\vdash\lambda\{ \ldots, l_i:x_i,\ldots, l_j:x_j=M_j, \ldots \}.M
    : \{ \ldots, l_i:t_i,\ldots, ?l_j:t_j,\ldots \} \rightarrow t}
\]
\end{block}

\end{frame}

\begin{frame}{Sous-typage et application partielle}

Pour \textbf{accorder réduction et typage},
une fonction de type $\tmabs{l_1:t_1,l_2:t_2}{t}$
doit pouvoir être considérée comme une fonction de type
$\tmabs{l_1:t_1}{\tmabs{l_2:t_2}{t}}$.

\vfill

\[
\inferrule{\mbox{$\Delta$ ineffaçable}}{
  \{ \Gamma, \Delta \}\rightarrow t \;{\color{red}\leq}\;
       \{\Gamma\}\rightarrow\{\Delta\}\rightarrow t
}
\qquad
\inferrule{\mbox{$\Delta$ effaçable}}{
  \{ \Gamma, \Delta \}\rightarrow t
 \;{\color{red}\leq}\;
  \{\Gamma\}\rightarrow t
}
\]

\vfill

Quelques équations invalides:
\begin{itemize}
\item $
\tmabs{l_1:t_1}{\tmabs{l_2:t_2}{t}}
\;{\color{red}\not\leq}\;
\tmabs{l_1:t_1,l_2:t_2}{t}
$
\item $\tmabs{\Gamma,?l:t,\Delta}{t'} \;{\color{red}\not\leq}\;
  \tmabs{\Gamma,l:t,\Delta}{t'}$
\end{itemize}
\end{frame}

%% =======================================================================

\begin{frame}{Typage du contenu}

%{\color{gray}[Version 1.0 beta]}
Chaque source a son propre type de contenu, \\
  un nombre quelconque de canaux audio, video ou midi.
%\begin{itemize}
%\item Contrôle des conversions stereo/mono/dolby
%\item Support plus naturel de la video et du MIDI
%\end{itemize}

\vfill

\begin{block}{Arités: fixes ou variables}
\[
a\quad ::=\quad \star \;|\; 0 \;|\; S(a)
\]
\vfill
\[
   \inferrule{~}{0\sub 0} \quad\quad
   \inferrule{A\sub A'}{S(A)\sub S(A')} \quad\quad
   \inferrule{~}{\star\sub\star} \quad\quad
   \inferrule{~}{0\sub \star} \quad\quad
   \inferrule{A\sub\star}{S(A)\sub \star}
\]\[
   \inferrule{A \sub A' \\ B\sub B' \\ C\sub C'}{(A,B,C) \sub (A',B',C')}
\]
\end{block}

\end{frame}

\begin{frame}{Typage du contenu}

\vspace{0.5cm}

  \texttt{\small
    \begin{tabular}{rcl}
    swap&:&(source(2,0,0)) -> source(2,0,0)\\
    blank &:& () -> source('*a,'*b,'*c) \\
    on\_metadata&:&(handler,source('*a,'*b,'*c)) -> \\
           & & source('*a,'*b,'*c)\\
    echo&:&(delay:float,source('\#a,0,0)) -> \\
        & & source('\#a,0,0)\\
    % DB vire le prefixe "video." pour faire de la place a gauche
    %  en vrai greyscale veut du fixed arity, pcq son implem est pas
    %  assez générale
    greyscale&:&(source('*a,'*b+1,'*c)) -> \\ & & source('*a,'*b+1,'*c)\\
    output.file &:& 
       (format('*a,'*b,'*c), string, \\
     & & \phantom{(}source('*a,'*b,'*c)) ->\\
     & & source('*a,'*b,'*c)
    \end{tabular}
  }

  \begin{block}{Remarques}
  \begin{itemize}
  \item Contraintes de type spécifiques
  \item On forcer les types inconnus
  \item Polymorphisme ad-hoc / non-paramétrique (FFI only)
  \end{itemize}
  \end{block}

\end{frame}

%% =======================================================================

\begin{frame}[fragile]{Horloges multiples}

%{\color{gray}[Version 1.0 beta]}
%Décentralisation de l'horloge

\[ \small
\xymatrix{
   *+[F]{\mathtt{playlist}}\ar[r]
   {\save "1,1"*+<10pt>[F--]\frm{} \restore}
   & *+[F]{\mathtt{crossfade}}\ar[r] &
      *+[F]{\mathtt{fallback}}\ar[r]& *+[F]{\mathtt{output}} \\
   & & *+[F]{\mathtt{jingles}}\ar[u]
   {\save "2,3"-<20pt>."1,4"."1,1"*+<20pt>[F--]\frm{} \restore}
   &
}
\]

\vfill

\begin{block}{Script}
Inférence automatique, mais pas statique
\begin{lstlisting}
output.icecast(%vorbis,mount="myradio",
  fallback([crossfade(playlist("some.txt")),jingles]))
\end{lstlisting}
\end{block}

\end{frame}

\begin{frame}{Horloges}

\begin{block}{Utilisations}
\begin{itemize}
\item \'Eviter les incohérences temporelles internes
   (\kw{cross}, \kw{stretch}\ldots)
\item Contrôler le parallélisme, exploiter plusieurs coeurs
\item Gérer les écarts et blocages de multiples horloges externes
\[
\xymatrix{
  \text{\only<1->{\color{red}}remote server A} \ar[dr] &
        & \mbox{\only<1->{\color{gray}}remote server B} \\
  \text{\only<1->{\color{blue}}soundcard 1} \ar[r]      &
                \text{liquidsoap} \ar[ur]\ar[r]\ar[dr]
        & \mbox{\only<1->{\color{blue}}soundcard 1}     \\
  \text{\only<1->{\color{green}}soundcard 2} \ar[ur] &
        & \mbox{\only<1->{\color{blue}}backup file}
}
\]
\end{itemize}
\end{block}

\end{frame}

%% =======================================================================

\begin{frame}{En pratique}

\begin{block}{Implémentation}
\begin{itemize}
 \item Langage: traits impératifs, suffisamment d'inférence
 \item User-friendly:
   messages simples, cacher ce qui peut l'être, doc
 \item OCaml: utilisation des étiquettes et de l'objet
 \item Structure par plugins
 \item Efficace: traitement par bloc, en place (analyse du partage)
\end{itemize}
\end{block}

\vfill

\begin{block}{Fonctionnalités}
\begin{itemize}
\item Sources: fichier, playlist, queues, script externe
\item Opérateurs: switch, add, traitement du signal, évènements
% TODO détection de blanc visible?
% \item Traitement du signal: compression, changer la hauteur, etc.
% \item Gestion d'évènements: pistes, metadonnée, silence
\item Interfaces: serveurs de diffusion, carte son, librairies audio
\item Formats: WAV, Ogg, MP3, AAC+, Flac, externe
\item Requêtes et protocoles, serveur, etc.
\end{itemize}
\end{block}

\end{frame}


%% =======================================================================

\begin{frame}{Conclusion}

\begin{block}{Liquidsoap}
\begin{itemize}
  \item Une boîte à outil pour le streaming multimedia
  \item Un logiciel OCaml relativement gros et grand public
  \item De la programmation fonctionnelle dans les mains de tous
\end{itemize}
\end{block}

\vfill

\begin{block}{Liquidsoap 2.0: un nouveau langage}
\begin{itemize}
\item Exposer la définition des sources dans le langage
\item Compiler: sûreté, efficacité, nouvelles applications
\item Liens avec les systèmes synchrones et hybrides
\end{itemize}
\end{block}

\end{frame}

\end{document}
