\documentclass{article}
\usepackage[utf8]{inputenc}
\usepackage[english]{babel}
\usepackage{amsfonts}
\usepackage{amsmath}
\usepackage{amsthm}
\usepackage{hyperref}
\usepackage[matrix,arrow]{xy}
% \usepackage{tikz}

% Rounded rectangle
% \tikzset{rr/.style={rectangle,thick,color=black}}

\newcommand{\TODO}[1]{\marginpar{\tiny #1}}

\newtheorem{theorem}{Theorem}
\newtheorem{definition}[theorem]{Definition}
\newtheorem{lemma}[theorem]{Lemma}
\newtheorem{property}[theorem]{Property}
\newtheorem{example}[theorem]{Example}
\newtheorem{remark}[theorem]{Remark}

\newcommand{\ce}{\mathop{\mathrm{e}}}
\newcommand{\dd}[1]{\mathop{\mathrm{d}#1}}
\newcommand{\N}{\mathbb{N}}
\newcommand{\R}{\mathbb{R}}
\newcommand{\intset}[1]{[#1]}
\renewcommand{\labelitemi}{--}
\newcommand{\resp}{resp.~}
\newcommand{\qtext}[1]{\quad\text{#1}\quad}
\newcommand{\qqtext}[1]{\qquad\text{#1}\qquad}
\newcommand{\qto}{\quad\to\quad}
\newcommand{\qcolon}{\quad:\quad}
\newcommand{\category}[1]{\mathbf{#1}}
\newcommand{\KN}{\category{KN}}

\newcommand{\ie}{i.e.~}

\title{Approximation of Continuous Kahn Networks}
\author{Samuel Mimram}

\hypersetup{
  pdftitle={\csname @title\endcsname},
  pdfauthor={\csname @author\endcsname},
  unicode=true,
  colorlinks=true,
  linkcolor=black,
  citecolor=black,
  urlcolor=black
}
\renewcommand{\C}{\mathcal{C}}

\begin{document}
\maketitle

We want something like functional reactive
programming~\cite{elliott1997functional}:
\begin{enumerate}
\item the underlying idealized model consists of streams of values indexed over
  a continuous time and events,
\item the actual implementation is discrete but should not explicitly depend on
  the sampling rate chosen.
\end{enumerate}
This allow nice things, for example we can choose other methods than Euler's to
solve involved ODE.

Continuous time in hybrid models~\cite{liu:causality-hybrid}.


\section{Kahn Networks}
\subsection{Formal definition}
We give an abstract definition of Kahn
networks~\cite{kahn:semantics-parallel}. There were a lot of investigations
about the discrete-time semantics of these
networks~\cite{hildebrandt2004relational, ...}.

A \emph{signature} $(\Sigma,\sigma,\tau)$ consists of a set~$\Sigma$ of
\emph{symbols} and two functions \hbox{$\sigma,\tau:\Sigma\to\N$}, which to
every symbol~$f$ associate its \emph{arity} and~\emph{coarity} respectively. We
often write~$f:m\to n$ to indicate that~$f$ is an operator such that~$\sigma(f)=m$
and \hbox{$\tau(f)=n$}.

Given an integer~$n\in\N$, we write~$\intset{n}=\{0,\ldots,n-1\}$ for the
canonical set with~$n$ elements.

The following definition is essentially the same as the notion of \emph{network}
introduced in~\cite{hasegawa2008finite}.\TODO{Check the differences. In
  particular, we should apparently allow ``floating ports'' so this messes up
  with output unicity.}
\begin{definition}
  A \emph{Kahn network} $K=(P,O,\lambda,s,t):m\to n$, with $m,n\in\N$, consists
  of
  \begin{itemize}
  \item a set~$P$ of \emph{ports},
  \item a finite set~$O$ of \emph{operators},
  \item a \emph{labeling function} $\lambda:O\to\Sigma$ which to every operator
    associates a symbol called its \emph{label},
  \item a \emph{source function}
    \[
    s\qcolon\intset{m}+\sum_{o\in O}\intset{\sigma(\lambda(o))}\qto P
    \]
  \item and a \emph{target function}
    \[
    t\qcolon \intset{n}+\sum_{o\in O}\intset{\tau(\lambda(o))}\qto P
    \]
  \end{itemize}
  where the sum above denotes the disjoint union of sets. We sometimes write
  $o:m\to n$ when $o$ is an operator labeled by~$f:m\to n$. Given an operator
  \hbox{$o:m\to n$} and an integer~$i$ such that $0\leq i<m$ (\resp $0\leq
  i<n$), we write~$o^i$ (\resp $o_i$) for the port which is the the image of
  $i\in\intset{\sigma(\lambda(o))}$ under~$s$ (\resp of
  $i\in\intset{\tau(\lambda(o))}$ under~$t$). We similarly write $K^i$ (\resp
  $K^j$) for the image under~$s$ (\resp $t$) of $i\in m$ (\resp $i\in n$). A
  Kahn network should satisfy the following condition:
  \begin{itemize}
  \item \emph{output unicity}: for every port~$p\in P$, there should be exactly
    one index~$i$ and operation~$o\in O$ such that
    \[
    \text{either } p=o_i \qqtext{or} p=K_i
    \]
  \end{itemize}
\end{definition}

\begin{remark}
  A Kahn network necessarily has a finite set of ports.
\end{remark}


For example, the echo process is....
% \[
% \begin{tikzpicture}
  % \node [rr] {+} (1,0);
  % \node [rr] {d} (1,1);
% \end{tikzpicture}
% \]

A morphism between two Kahn networks is...

Given a signature~$\Sigma=(\Sigma,\sigma,\tau)$, we write~$\KN_\Sigma$ for the
category whose objects are natural integers and whose morphisms are Kahn
networks \hbox{$K:m\to n$}. The composite of two networks $K:m\to n$ and~$L:n\to
p$ is the network $L\circ K:m\to p$ defined by
\begin{itemize}
\item the set of ports is defined as the pushout
  \[
  \xymatrix{
    &\ar[dl]_{\tau_K}\intset{n}\ar[dr]^{\sigma_L}&\\
    P_K\ar@{.>}[dr]&&\ar@{.>}[dl]P_L\\
    &P_{L\circ K}&\\
  }
  \]
\item the set of operators is $O_{L\circ K}=O_L+O_K$ with~$\lambda_{L\circ
    K}=\lambda_L+\lambda_K$ as labeling function,
\item the source and target functions are obtained similarly from the source and
  target functions of~$K$ and~$L$.
\end{itemize}
The identity network on an object~$n$ is the network whose set of ports
is~$\intset{n}$, set of operators is empty, labeling function is the initial
arrow and source and target functions are identity $\intset{n}\to\intset{n}$.

We have actually constructed a bicategory since composition is defined up to
isomorphism. However, we can recover a category by considering KN up to iso of
KN.

This construction can be extended without difficulty to typed Kahn networks. A
nice and abstract description of this construction can be carried on using
polygraphs~\cite{burroni1993higher}, but we won't go into this here.

\subsection{Free traced monoidal categories}
Recall that a traced symmetric monoidal
category~\cite{joyal-street-verity:traced-monoidal-categories} is...

Given a signature~$\Sigma$, a $\Sigma$-object in a monoidal category~$\C$
is... The free category on a signature~$\Sigma$ is...

\begin{theorem}
  Given a signature~$\Sigma$, the category~$\KN_\Sigma$ is the free traced
  cartesian category on~$\Sigma$.\TODO{Be careful of floating ports, we might
    have to add an axiom.}
\end{theorem}

An explicit description of the free traced symmetric monoidal category on a
category is given in~\cite{abramsky:traced-compact-closed} and generalized
in~\cite{hasegawa2008finite}. We recover these construction (the first one is
the special case where all the symbols have arity and coarity~$1$).

Notes:
\begin{itemize}
\item A traced cartesian category is the same as a cartesian category with a
  fixpoint operator~\cite{hasegawa1997recursion}.
\item If we only consider a delayed trace, then all the axioms of traced SMC are
  satisfied, \emph{excepting} the yanking axiom. Actually, it seems that the
  notion of traced monoidal category without yanking has been investigated under
  the name of \emph{categories with feedback}~\cite{katis2002feedback}, with
  this idea in mind. They even describe the free category with feedback on a
  category. Some links between this notion and electronic circuits are
  investigated in~\cite{katis1999algebra}.
\item The property of a circuit not to have instant feedback is linked to
  \emph{feedback reliability} in~\cite{pardo2004synchronous}, we shall restrict
  to those. Can this be enforced by a typing system?
\end{itemize}

\subsection{Kahn networks with delay}
A \emph{Kahn network with delay} on a signature sigma is a Kahn network of the
signature~$\Sigma\uplus\{d\}$ where $d:1\to 1$ is an symbol called
\emph{delay}. Such a Kahn network is \emph{instant-feedback free} when TODO all
the cycles contain at least a delay.

\begin{theorem}
  The category of feedback-free Kahn networks with delay is the free feedback
  category on~$\Sigma$.
\end{theorem}


\section{Semantics of Kahn networks}
\subsection{Discrete-time semantics}
Recall the usual discrete time semantics.

\begin{theorem}
  The discrete-time semantics can be recovered from the continuous time by
  considering the feedback category of partially defined signals where delays
  are angelic: they delay in a way such that we get the most information where
  the information order on a sequence is such that~$s\sqsubseteq t$
  whenever~$\tilde{s}$ is a prefix of~$\tilde{t}$ where~$\tilde{s}$ is the
  sequence obtained from~$s$ by removing undefined values (TODO or something
  like that).
\end{theorem}

\subsection{Continuous-time semantics}
Since the category~$\KN_\Sigma$ is the free trace cartesian category
on~$\Sigma$, given any traced cartesian category~$\C$, together with an
interpretation of the symbols into this category, an interpretation
functor~$\KN_\Sigma\to\C$ is defined by the universal property... In other
words, we ``only'' have to define a proper traced cartesian category~$\C$ in
order to give a semantics to our networks. Actually, this idea of having a
traced monoidal category with continuous functions is quite natural (for example
the possibility of constructing a traced monoidal category of manifolds and
smooth maps is evoked in~\cite{abramsky1996retracing}), but has not been
formally investigated to the best of our knowledge.

Since we don't mind about types, we might assume that the objects of~$\C$ are
natural integers in~$\N$. A morphism~$f:m\to n$ should be ``something like'' a
differentiable function~$f:\R^m\to\R^n$. Notice that we surely want to have
piecewise differentiable functions and act as if they were differentiable. For
example, we want to have the square waveform, or more simply the \emph{Heaviside
  function}~$H$ defined by
\[
H(t)=
\begin{cases}
  0&\text{if $t<0$}\\
  1&\text{if $t\geq 0$}\\
\end{cases}
\]
This function obviously cannot be derived, but we can achieve what we want if we
consider distributions instead of functions: the derivative is the Dirac
``function''. Do we really want to go into this?... Actually we don't seem to
have, see for example Carathéodory's existence theorem for
ODE\footnote{\url{http://en.wikipedia.org/wiki/Carath\%C3\%A9odory\%27s_existence_theorem}}. A
whole theory has been developed around this, calling this \emph{impulsive
  differential equations}~\cite{lakshmikantham1989theory}. Do we want to
consider Diracs as events? The derivative of the square function is a sequence
of Diracs alternatively up and down; however, we really want to differentiate it
in order for example to apply a first-order filter on it. What does all this
means in details?

We shall also avoid the ``Zeno effect'': first smooth maps and then extend to
piecewise smooth maps?

Interestingly, the Cauchy-Lipschitz theorem (called the Picard-Lindelöf theorem
everywhere but in
France)\footnote{\url{http://en.wikipedia.org/wiki/Picard\%E2\%80\%93Lindel\%C3\%B6f_theorem}}
uses the Banach fixpoint theorem in order to show the existence and unicity of
solutions to some ODE. Could this be related to the usual fixpoint semantics
of KN? The theorem says that

\begin{theorem}
  Given an ODE $y'(t)=f(t,y(t))$, with
  \[
  f:[t_0-\alpha,t_0+\alpha]\times[a-\beta,a+\beta]^n\to\R^n
  \]
  continuous and Lipschitz continuous in~$y$, with Lipschitz constant~$L$ for
  every \hbox{$t\in[t_0-\alpha,t_0+\alpha]$}, and with initial
  condition~$y(t_0)=y_0$.  Then there exists a unique solution on the interval
  $[t_0-b,t_0+b]$ with~$b=\min\{\alpha,\beta/M\}$.
\end{theorem}
\begin{proof}
  The sequence of functions
  \[
  y_{n+1}(t)=y_0+\int_{t_0}^tf(t,y_n(t))\dd t
  \]
  converges uniformly to the solution.
\end{proof}

Morally, the hypothesis of this theorem should be satisfied in our KN. Can we
enforce this by typing or static analysis? A nice course on
ODE\footnote{\url{http://www.math.byu.edu/~grant/courses/m634/f99/lectures.html}}.

\section{Discrete-time approximations}
We write~$T$ for the sampling period and~$t_n=n\times T$. The idea of the
\emph{Euler method} is to approximate~$y'(t_{n+1})$ by
\[
\frac{y(t_{n+1})-y(t_n)}n
\]

By Taylor's theorem, there exists~$\xi_n\in[t_n,t_{n+1}]$ such that
\[
y(t_{n+1})=y(t_n)+Ty'(t_n)+\frac{T^2}2 y''(\xi_n)
\]
Now, if $y$ satisfies the equation $y'(t)=f(t,y(t))$, we have
\[
y(t_{n+1})=y(t_n)+Tf(t_n,y(t_n))+\frac{T^2}2 y''(\xi_n)
\]
The \emph{single-step truncation error} associated with the Euler method is thus
\[
e_{n+1}=\frac{T^2}2 y''(\xi_n)
\]
By generalizing this in order to take in account the \emph{cumulative error}, it
can be
shown\footnote{\url{http://www.mathcs.emory.edu/ccs/ccs315/ccs315/node25.html}}
that

\begin{theorem}
  Suppose that
  \[
  M=\sup_{s\in\R}\left|\frac{\partial f(s,z)}{\partial z}\right|<\infty
  \]
  Then we have the following error bound
  \[
  |y(t_n)-\tilde{y_n}|<\ce^{(t_n-t_0)M}|y(t_0)-\tilde{y_0}|+T\left(\frac{\ce^{(t_n-t_0)M}-1}{2M}\right)\max_{t_0\leq t\leq t_n}|y''(t)|
  \]
  where $\tilde{y_n}$ denotes the value computed at step~$n$ by the Euler
  method. In particular, if $\tilde{y_0}=y(t_0)$ and $y''$ is bounded then
  \[
  |y(t_n)-\tilde{y_n}|<c_nT
  \]
  for some constant~$c_n$. Moreover, this can be generalized to any ODE.
\end{theorem}

\begin{remark}
  Faster convergences can be obtained using Taylor series or Runge-Kutta
  methods. For example, with RK4 the cumulative error is $O(T^4)$.
\end{remark}

\section{Future works}
\begin{itemize}
\item Link independent subcomponents of the networks with Eigenspace
  decomposition linear ODE.
\end{itemize}

\bibliographystyle{alpha}
\bibliography{sound}
\end{document}
