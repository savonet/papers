\documentclass{article}
\usepackage[utf8]{inputenc}
\usepackage[english]{babel}
\usepackage{amsfonts}
\usepackage{amsmath}
\usepackage{amsthm}
\usepackage{hyperref}
\usepackage[matrix,arrow]{xy}
% \usepackage{tikz}

% Rounded rectangle
% \tikzset{rr/.style={rectangle,thick,color=black}}

\newtheorem{theorem}{Theorem}
\newtheorem{lemma}[theorem]{Lemma}
\newtheorem{property}[theorem]{Property}
\newtheorem{example}[theorem]{Example}
\newtheorem{remark}[theorem]{Remark}

\newcommand{\N}{\mathbb{N}}
\newcommand{\R}{\mathbb{R}}
\newcommand{\intset}[1]{[#1]}
\renewcommand{\labelitemi}{--}
\newcommand{\resp}{resp.~}
\newcommand{\qtext}[1]{\quad\text{#1}\quad}
\newcommand{\qqtext}[1]{\qquad\text{#1}\qquad}
\newcommand{\qto}{\quad\to\quad}
\newcommand{\qcolon}{\quad:\quad}
\newcommand{\category}[1]{\mathbf{#1}}
\newcommand{\KN}{\category{KN}}

\title{Approximation of Continuous Kahn Networks}
\author{Samuel Mimram}

\hypersetup{
  pdftitle={\csname @title\endcsname},
  pdfauthor={\csname @author\endcsname},
  unicode=true,
  colorlinks=true,
  linkcolor=black,
  citecolor=black,
  urlcolor=black
}

\begin{document}
\maketitle

\section{Kahn Networks}
\subsection{Formal definition}
A \emph{signature} $(\Sigma,\sigma,\tau)$ consists of a set~$\Sigma$ of
\emph{symbols} and two functions \hbox{$\sigma,\tau:\Sigma\to\N$}, which to
every symbol~$f$ associate its \emph{arity} and~\emph{coarity} respectively. We
often write~$f:m\to n$ to indicate that~$f$ is an operator such that~$\sigma(f)=m$
and \hbox{$\tau(f)=n$}.

Given an integer~$n\in\N$, we write~$\intset{n}=\{0,\ldots,n-1\}$ for the
canonical set with~$n$ elements.

A \emph{Kahn network} $K=(P,O,\lambda,s,t):m\to n$, with $m,n\in\N$, consists of
\begin{itemize}
\item a set~$P$ of \emph{ports},
\item a set~$O$ of \emph{operators},
\item a \emph{labeling function} $\lambda:O\to\Sigma$ which to every operator
  associates a symbol called its \emph{label},
\item a \emph{source function}
  \[
  s\qcolon\intset{m}+\sum_{o\in O}\intset{\sigma(\lambda(o))}\qto P
  \]
\item and a \emph{target function}
  \[
  t\qcolon \intset{n}+\sum_{o\in O}\intset{\tau(\lambda(o))}\qto P
  \]
\end{itemize}
where the sum above denotes the disjoint union of sets. We sometimes write
$o:m\to n$ when $o$ is an operator labeled by~$f:m\to n$. Given an operator
\hbox{$o:m\to n$} and an integer~$i$ such that $0\leq i<m$ (\resp $0\leq i<n$),
we write~$o^i$ (\resp $o_i$) for the port which is the the image of
$i\in\intset{\sigma(\lambda(o))}$ under~$s$ (\resp of
$i\in\intset{\tau(\lambda(o))}$ under~$t$). We similarly write $K^i$ (\resp
$K^j$) for the image under~$s$ (\resp $t$) of $i\in m$ (\resp $i\in n$). A Kahn
network should satisfy the following condition:
\begin{itemize}
\item \emph{output unicity}: for every port~$p\in P$, there should be exactly
  one index~$i$ and operation~$o\in O$ such that
  \[
  \text{either } p=o_i \qqtext{or} p=K_i
  \]
\end{itemize}

For example, the echo process is....
% \[
% \begin{tikzpicture}
  % \node [rr] {+} (1,0);
  % \node [rr] {d} (1,1);
% \end{tikzpicture}
% \]

Given a signature~$\Sigma=(\Sigma,\sigma,\tau)$, we write~$\KN_\Sigma$ for the
category whose objects are natural integers and whose morphisms are Kahn
networks \hbox{$K:m\to n$}. The composite of two networks $K:m\to n$ and~$L:n\to
p$ is the network $L\circ K:m\to p$ defined by
\begin{itemize}
\item the set of ports is defined as the pushout
  \[
  \xymatrix{
    &\ar[dl]_{\tau_K}\intset{n}\ar[dr]^{\sigma_L}&\\
    P_K\ar@{.>}[dr]&&\ar@{.>}[dl]P_L\\
    &P_{L\circ K}&\\
  }
  \]
\item the set of operators is $O_{L\circ K}=O_L+O_K$ with~$\lambda_{L\circ
    K}=\lambda_L+\lambda_K$ as labeling function,
\item the source and target functions are obtained similarly from the source and
  target functions of~$K$ and~$L$.
\end{itemize}
The identity network on an object~$n$ is the network whose set of ports
is~$\intset{n}$, set of operators is empty, labeling function is the initial
arrow and source and target functions are identity $\intset{n}\to\intset{n}$.

\subsection{Free compact closed categories}
\begin{theorem}
  Given a signature~$\Sigma$, the category~$\KN_\Sigma$ is a full subcategory
  of the free cartesian compact closed category on~$\Sigma$.
\end{theorem}

\subsection{Semantics of Kahn networks}
To every process $K:m\to n$ one associates a partial function~$\R^m\to\R^n$
defined inductively as follows.
\end{document}
