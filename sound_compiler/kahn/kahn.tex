\documentclass{article}
\usepackage[utf8]{inputenc}
\usepackage[english]{babel}
\usepackage{amsfonts}
\usepackage{amsmath}
\usepackage{hyperref}

% \newcommand{\Types}{Types}
% \newcommand{\Ops}{Ops}
\newcommand{\N}{\mathbb{N}}
\newcommand{\intset}[1]{[#1]}
\renewcommand{\labelitemi}{--}
\newcommand{\resp}{resp.~}
\newcommand{\qtext}[1]{\quad\text{#1}\quad}

\title{Approximation of Continuous Kahn Networks}
\author{Samuel Mimram}

\hypersetup{
  pdftitle={\csname @title\endcsname},
  pdfauthor={\csname @author\endcsname},
  unicode=true,
  colorlinks=true,
  linkcolor=black,
  citecolor=black,
  urlcolor=black
}

\begin{document}
\maketitle

\section{Kahn Networks}
A \emph{signature} $(\Sigma,\sigma,\tau)$ consists of a set~$\Sigma$ of
\emph{symbols} and two functions \hbox{$\sigma,\tau:\Sigma\to\N$}, which to
every symbol~$f$ associate its \emph{arity} and~\emph{coarity} respectively. We
often write~$f:m\to n$ to indicate that~$f$ is an operator such that~$\sigma(f)=m$
and \hbox{$\tau(f)=n$}.

Given an integer~$n\in\N$, we write~$\intset{n}=\{0,\ldots,n-1\}$ for the
canonical set with~$n$ elements.

A \emph{Kahn network} $K=(P,O,\lambda,s,t):m\to n$, with $m,n\in\N$, consists of
\begin{itemize}
\item a set~$P$ of \emph{ports},
\item a set~$O$ of \emph{operators},
\item a \emph{labeling function} $\lambda:O\to\Sigma$ which to every operator
  associates a symbol called its \emph{label},
\item a \emph{source function}
  \[
  \intset{m}+\sum_{o\in O}\intset{\sigma(\lambda(o))}\to P
  \]
\item and a \emph{target function}
  \[
  \intset{n}+\sum_{o\in O}\intset{\tau(\lambda(o))}\to P
  \]
\end{itemize}
where the sum above denotes the disjoint union of sets. We sometimes write
$o:m\to n$ when $o$ is an operator labeled by~$f:m\to n$. Given an operator
\hbox{$o:m\to n$} and an integer~$i$ such that $0\leq i<m$ (\resp $0\leq i<n$),
we write~$o^i$ (\resp $o_i$) for the port which is the the image of
$i\in\intset{\sigma(\lambda(o))}$ under~$s$ (\resp of
$i\in\intset{\tau(\lambda(o))}$ under~$t$). We similarly write $K^i$ (\resp
$K^j$) for the image under~$s$ (\resp $t$) of $i\in m$ (\resp $i\in n$). A Kahn
network should satisfy the following condition:
\begin{itemize}
\item \emph{output unicity}: for every port~$p\in P$, there should be exactly
  one index~$i$ and operation~$o\in O$ such that
  \[
  \text{either } p=o_i \qtext{or} p=K_i
  \]
\end{itemize}


\end{document}
