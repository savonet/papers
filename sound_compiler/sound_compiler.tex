\documentclass[a4paper,titlepage]{article}
\usepackage[utf8]{inputenc}
\usepackage[english]{babel}
\usepackage{amsmath}
\usepackage{amssymb}
\usepackage{amsthm}
\usepackage{mathpartir}
\usepackage{hyperref}
\usepackage{rotating}

\renewcommand{\labelitemi}{--}
\newcommand{\ie}{i.e.~}

\newcommand{\Var}{Var}
\newcommand{\ATypes}{ATypes}
\newcommand{\ret}{\mathop{\_\hspace{-.3ex}\_}}
\newcommand{\ot}{\leftarrow}
\newcommand{\oT}{\Leftarrow}
\newcommand{\regle}[1]{\mathrm{[#1]}}

\newtheorem{theorem}{Theorem}
\newtheorem{lemma}[theorem]{Lemma}
\newtheorem{property}[theorem]{Property}
\newtheorem{example}[theorem]{Example}
\newtheorem{remark}[theorem]{Remark}

\newcommand{\TODO}[1]{\marginpar{\footnotesize #1}}

\title{Ideas for \\An Optimizing Compiler \\for Sound Analysis and Synthesis}
\author{Samuel Mimram}

\hypersetup{
  pdftitle={\csname @title\endcsname},
  pdfauthor={\csname @author\endcsname},
  unicode=true,
  colorlinks=true,
  linkcolor=black,
  citecolor=black,
  urlcolor=black
}
\renewcommand{\C}{\mathcal{C}}

\begin{document}
\maketitle
\tableofcontents
\newpage

\section{What we want}
\subsection{What we should be able to do}
\begin{itemize}
\item We want to be able to manipulate streams of data: audio (streams of
  samples), video (streams of images), etc.
\item We should distinguish between two kind of things that we manipulate:
  \emph{values} which are always available in a stream (samples in audio data)
  and \emph{events} which are punctual data (metadata, notes, etc.). For example
  the type for metadata is \verb|(string * string) event|.
\item We should be able to trigger a event of type unit when a data changes
  (typically when the audio data gets louder than a certain value): we need a
  function \verb|bool -> bool event| which triggers an event with the new value
  when the value has changed.
\item How do we use this function to program piecewise linear functions
  (typically ADSR envelopes)? We also need to be able to consider \emph{relative
    times} for this (decay is after beginning of attack + 20ms).
\item We should moreover be able to deal with \emph{partial streams} ie streams
  where data is not always available.
\item It would be very nice to have a good typing of data represented in
  frequency domain (that is after an FFT): how do we integrate this properly in
  the language.
\item Have a good notion of track. We should be able to use events for this.
\item Images (yes images, not video streams) bear many similarities with
  streams: they can be seen as stream indexed by two finite clocks (the
  coordinates). Now, we would like to do the same kind of optimisations eg
  \verb|rotate(greyscale(img))| needs two loops over pixels but it could be done
  using only one loop! So we need to have multiple clocks.
\item Allow changing the clock rate: to have an acceleration operator or to
  resample.
\item Be able to merge events ie \verb|'a event -> 'a event -> 'a events|. So we
  should have biproducts for (\verb|unit|) events!
\end{itemize}

\subsection{What the language should look like}
\begin{itemize}
\item Operators with multiple outputs as well as multiple inputs. We really live
  in a monoidal category with biproducts.
\item What is a good syntax for this? We could have a categorical syntax (à la
  Faust) but having the opportunity to name wires seems to be nice, so both
  would be perfect (eg in $\lambda$-calculus we can do both $g\circ f$ and
  $\lambda x.g(f(x))$).
\item Typing system. We should begin with something simple.
\item We need a trace operator: instant feedback should be ruled out by either
  typing or checks for acyclicity (however feedbacks are legal in presence of a
  delay).
\item How do we type clocks properly?
\item The language should be high-level and implementation-independent. In
  particular, it would be nice to be able to compile audio algos both to int and
  float and compare the difference of performance.
\item Maybe do we need some limited form of dependent types. For example, to
  implement a function
  \begin{center}
    \texttt{map : $n$:int -> ('a -> 'a) -> ('a$\null^n$ -> 'a$\null^n$)}
  \end{center}
   which would apply a same function to every channel of a stream.
 \item Overloading for events / values? eg \texttt{+} can be given both types
   \begin{center}
     \verb|+ : int -> int|
     \qquad and\qquad
     \verb|+ : int event -> int event|
   \end{center}
   Do we want this for all operators?
\end{itemize}

\subsection{Further optimizations}
\begin{itemize}
\item We should implement all the classical optimizations: constant propagation,
  expression factorization, etc.
\item Some optims are also specific to the domain of application: merge for
  loops as much as possible, etc.
\item Since the code will be very specific, we can vectorize it easily and use
  SIMD operations. Even better, we can think of using very funny things such as
  compiling to graphic cards (Cuda or OpenCL) or VLSI.
\item We should distinguish between three kind of values.
  \begin{enumerate}
  \item \emph{constant values}: never change during time
  \item \emph{sparse values}: rarely change during time. In practice, this means
    that we can consider them as constant during a buffer. The typical example
    is a value read from an user input such as a GUI.
  \item \emph{values} which are neither constant or sparse.
  \end{enumerate}
  The constant values might be inferred automatically. However, sparse value
  might require annotations. Optimize computations on constant and sparse
  streams, ie compute functions only once.
\item Infer which functions have side effects: for these we cannot compute once
  functions on constant streams.
\item The stream data might be a simple value (floats for audio) or an allocated
  value (images). In the last case, we should avoid copies and we need to do
  this in a clever way.
\end{itemize}

\section{The language}
\subsection{Terms}
Suppose given a set~$\Var$ of \emph{variables}. The syntax for terms~$M$ is
\[
M
\qquad::=\qquad
x
\qquad|\qquad
\lambda x.M
\qquad|\qquad
MM
\qquad|\qquad
\ret
\]
We sometimes write $\lambda xy.M$ instead of $\lambda x.\lambda
y.M$. Substitution of a variable $x$ by a term~$N$ in a term~$M$,
written~$M[N/x]$ is defined inductively by
\[
\begin{array}{c}
  x[N/x]=N
  \qquad
  y[N/x]=y
  \qquad
  (\lambda y.M)[N/x]=\lambda y.M[N/x]
  \\
  (M_1M_2)[N/x]=(M_1[N/x])(M_2[N/x])
  \qquad
  \ret[N/x]=\ret
\end{array}
\]
where~$y\neq x$. Terms are considered modulo $\alpha$-equivalence.

The reductions rules are
\[
\begin{array}{r@{\quad}c@{\quad}l}
  (\lambda x.M)N&\longrightarrow&M[N/x]\\
  M(\ret M_1\ldots M_k)&\longrightarrow&MM_1\ldots M_k\\
\end{array}
\]
So, the new constructor $\ret$ should be seen as a \emph{return} operator which
might return multiple results.

\begin{example}
  Here you go:
  \begin{itemize}
  \item the duplicator: $\delta=\lambda x.(\ret xx)$
  \item the erasor: $\lambda x.0$
  \item composite of the two:
    \[
    \lambda y.(\lambda x.0)((\lambda x.\ret xx)y)
    \quad\longrightarrow\quad
    \lambda y.\lambda(x.0)(\ret yy)
    \quad\longrightarrow\quad
    \lambda y.\lambda(x.0)yy
    \quad\longrightarrow\quad
    \lambda y.0y
    \quad=\quad
    \lambda y.y
    \]
  \item the coidentity: $\ret$
    \[
    f\ret\quad\longrightarrow\quad f
    \]
  \item the $\eta$-coexpansion: $\lambda x.\ret x$
    \[
    (\lambda x.\ret x)f\quad\longrightarrow\quad fx
    \]
  \item similarly to composition $g\circ f=\lambda x.f(g(x))$, cocomposition is
    defined by $g\circ \lambda z_i.M=\lambda z_i.g M$. Ex: duplicate audio in an
    audio + video stream
    \[
    (\lambda x.\ret x x)\circ(\lambda z.\ret a v)
    =
    \lambda z.(\lambda x.\ret xx)(\ret av)
    \quad\longrightarrow\quad
    \lambda z.(\lambda x.\ret xx)av
    \quad\longrightarrow\quad
    \lambda z.\ret aav
    \]
  \end{itemize}
\end{example}

Of course, there is an obvious family of critical pairs in our system:
\[
M[\ret M_1\ldots M_k/x]
\quad\longleftarrow\quad
(\lambda x.M)(\ret M_1\ldots M_k)
\quad\longrightarrow\quad
(\lambda x.M)M_1\ldots M_k
\]
Which branch of the critical pair is chosen will depend on typing, which is thus
crucial in our system.

\subsection{Types}
Suppose given a set $\ATypes$ of \emph{atomic types}. The syntax for types~$T$ is
\[
T
\qquad::=\qquad
A
\qquad|\qquad
L\to L
\qquad|\qquad
T \times T
\qquad|\qquad
1
\]
where~$A\in\ATypes$ and~$L$ is a list of types. A list of types is written
\hbox{$L=(A_1,\ldots,A_n)$} and concatenation of two lists~$L_1$ and~$L_2$ is
written~$L_1\cdot L_2$. A term of type~$A$ is called a \emph{value}. We
sometimes write~$L^*$ instead of $L\to()$ and a term of this type called a
\emph{covalue}.

TODO: define sequents, which are considered modulo commutation in~$\Gamma$

\[
\begin{array}{cc}
  \inferrule{\null}
  {\Gamma,x:A\vdash x:A}{\regle{Ax}}
  &
  \inferrule
  {\null}
  {\Gamma\vdash \ret :()\to()}{\regle{Ret}}
  \\[4ex]
  \inferrule
  {\Gamma,x:A\vdash M:K\to L}
  {\Gamma\vdash \lambda x.M:(A)\cdot K\to L}{\regle{Abs}}
  &
  \inferrule
  {\Gamma\vdash M:()\to L\\\Gamma\vdash N:A}
  {\Gamma\vdash MN:()\to L\cdot(A)}{\regle{Ans}}
  \\[4ex]
  \inferrule
  {\Gamma\vdash M:(A)\cdot K\to L\\\Gamma\vdash N:A}
  {\Gamma\vdash MN:K\to L}{\regle{App}}
  &
  \inferrule
  {\Gamma\vdash M:K_1\to L\\\Gamma\vdash N:()\to K_1\cdot K_2}
  {\Gamma\vdash MN:()\to L\cdot K_2}{\regle{App'}}
  \\[4ex]
  \inferrule
  {\Gamma\vdash M:()\to(A)}
  {\Gamma\vdash (\lambda x.x)M:A}{\regle{Ret'}}
\end{array}
\]
\TODO{We need a rule to go from \hbox{$()\to(A)$} to $A$}. The rules in the left
column are traditional rules in $\lambda$-calculus, with application
$\regle{App}$ reformulated to our way of writing types of functions with
multiple arguments. The rule $\regle{Ret}$ types the return statement (used for
introducing arrow types), the rule $\regle{Ans}$ adds a new result to the
current function and the rule~$\regle{App'}$ applies a function to some
results. The restriction on the $\regle{Ans}$ rule is to avoid an ambiguity
with~$\regle{App'}$.

\begin{remark}
  There are four rules for application, so we should study conflicts between
  rules.
  \begin{itemize}
  \item $\regle{Ans}$ vs. $\regle{App}$: there is no conflict between these two
    rules since~$M$ is nullary in the first one and non-nullary in the second
    one
  \item $\regle{Ans}$ vs. $\regle{App'}$:
    \[
    \inferrule{
      \Gamma\vdash M:()\to L
      \\
      \Gamma\vdash N:()\to K
    }
    {
      \Gamma\vdash MN:()\to L\cdot(()\to K)
      \\\\\text{or}\\\\
      \Gamma\vdash MN:()\to K
    }
    \]
    Two side conditions are possible to avoid that:
    \begin{enumerate}
    \item in $\regle{Ans}$: $A$ is not a nullary arrow type
    \item in $\regle{App'}$: $K_1\neq ()$
    \end{enumerate}
    The first one seems to be preferable since we want to be able to insert new
    results into functions. However, it removes the possibility to return a
    stream\ldots We could also distinguish between the two kinds of composition,
    but this looks like an overkill just for this case.
  \item $\regle{App}$ vs. $\regle{App'}$: unification gives $K_1=(A)\cdot K$ and
    $A=()\to K_1\cdot K_2$ which has no solution (there are no recursive types)
  \item $\regle{Ret'}$ vs. \ldots: this one is really particular and we can add
    a constant for an operator behaving like identity (but different from
    identity during typing so that we identify this case):
    \[
    !\quad:\quad(()\to(A))\to A
    \]
    with the reduction rule
    \[
    !M
    \qquad\longrightarrow\qquad
    (\lambda x.x)M
    \]
  \end{itemize}
\end{remark}

\begin{example}
  Here you go:
  \begin{itemize}
  \item the duplicator
    \[
    \inferrule{
      \inferrule{
        \inferrule{
          \inferrule{\null}{\Gamma\vdash\ret:()\to()}{\regle{Ret}}
          \\
          \inferrule{\null}{\Gamma,x:A\vdash x:A}{\regle{Ax}}
        }
        {\Gamma,x:A\vdash \ret x:()\to(A)}
        \\
        \inferrule{\null}{\Gamma,x:A\vdash x:A}{\regle{Ax}}
      }
      {\Gamma,x:A\vdash\ret xx:()\to(A,A)}{\regle{Ans}}
    }
    {\Gamma\vdash\lambda x.\ret xx:(A)\to(A,A)}{\regle{Abs}}
    \]
  \item an audio + video stream
    \[
    \tiny
    \inferrule{
      \inferrule{
        \inferrule{\null}{\Gamma,a:A,v:V\vdash\ret:()\to()}{\regle{Ret}}
        \\
        \inferrule{\null}{\Gamma,a:A,v:V\vdash a:A}{\regle{Ax}}
      }
      {\Gamma,a:A,v:V\vdash \ret a:()\to(A)}{\regle{Ans}}
      \\
      \inferrule{\null}{\Gamma,a:A,v:V\vdash v:V}{\regle{Ax}}
    }
    {\Gamma,a:A,v:V\vdash\ret av:()\to(A,V)}{\regle{Ans}}
    \]
  \item duplicate audio in an audio + video stream
    \[
    \tiny
    \inferrule{
      \inferrule{\vdots}{\Gamma,a:A,v:V\vdash \lambda x.\ret xx:(A)\to(A,A)}
      \\
      \inferrule{\vdots}{\Gamma,a:A,v:V\vdash \ret av:()\to(A,V)}
    }
    {\Gamma,a:A,v:V\vdash(\lambda x.\ret xx)(\ret av):()\to(A,A,V)}{\regle{App'}}
    \]
  \end{itemize}
\end{example}

\begin{lemma}[Substitution]
  If~$\Gamma,x:A\vdash M:B$ and~$\Gamma\vdash N:A$ then $\Gamma\vdash M[N/x]:B$.
\end{lemma}
\begin{proof}
  TODO
\end{proof}

\begin{theorem}[Subject reduction]
  If $\Gamma\vdash M:A$ and $M\longrightarrow M'$ then~$\Gamma\vdash M':A$.
\end{theorem}
\begin{proof}
  TODO using substitution lemma
\end{proof}

\begin{theorem}[Cut-elimination]
  TODO
\end{theorem}
\begin{proof}
  TODO
\end{proof}

\begin{theorem}[Unicity of typing]
  TODO
\end{theorem}
\begin{proof}
  TODO
\end{proof}

\begin{theorem}[Strong normalization]
  Typable terms are strongly normalizing.
\end{theorem}
\begin{proof}
  TODO
\end{proof}

The big theorem that we really want is the following one. This is close to the work of~\cite{hasegawa:sharing-graphs}
\begin{theorem}
  When we only have functions of type~$K\to L$ with~$K$ and~$L$ containing only
  base types (no arrow types) we get the free ``boxes and wires'' category.
\end{theorem}

\subsection{Categorical models}
It is natural to want our category to be cartesian coclosed. However this won't
work because
\begin{property}
  Any cartesian coclosed~$\C$ is equivalent to the terminal
  category\footnote{\url{http://ncatlab.org/nlab/show/cocartesian+closed+category}}.
\end{property}
\begin{proof}
  Suppose that~$\C$ is cartesian coclosed. We write $B\oT A$ for the left
  adjoint to the product, ie
  \[
  \inferrule
  {A\vdash B\times C}
  {A\oT B\vdash C}
  \]
  Since the category~$\C$ is cartesian, it has a terminal object~$1$ (so it is
  in particular non-empty). Now, given two objects~$A$ and~$B$, we have the
  following natural bijections between hom-sets
  \[
  \inferrule{
    \inferrule{
      A\vdash B
    }
    {A\vdash B\times 1}
  }
  {A\oT B\vdash 1}
  \]
  Since~$1$ is terminal, the hom-set $\C(A,B)$ is thus reduced to one element.
\end{proof}

Maybe we can grasp ideas from~\cite{faure-miquel:cat-par-lambda} who also a have
a parallel construction (which is different).

\subsection{Type inference}
TODO

\section{Hot topics}
\subsection{A syntax with coarities and cocurryfication}
We want operators with coarities e.g. the operator \texttt{swap}, which
exchanges the two channels of a stereo stream, has two inputs of type
\texttt{audio} and two outputs of type \texttt{audio}\TODO{Actually, it might be
  more natural to give it a type with one input and one output both of type
  \texttt{(audio * audio)}, but anyway.}. We write its type as follows:
\begin{center}
  \verb|swap : (audio , audio) -> (audio , audio)|
\end{center}
The language should support currying, so \texttt{swap} can be transformed into
\begin{center}
  \verb|swap' : (audio * audio) -> (audio , audio)|
\end{center}
But the language should also support \emph{cocurrying}, so that \texttt{swap}
can also be transformed into
\begin{center}
  \verb|swap'' : (audio , audio) -> (audio * audio)|
\end{center}
Similarly to the usual arrow notation, a list \verb|(A,B,C,D)| on the left of an
arrow should be read as \verb|(A,(B,(C,(D))))|. We often write \verb|A| instead
of \verb|(A)|.

As usual, applications can be partial: from \verb|f : (A,B)->C| and \verb|a : A|
we can construct \verb|(f a) : B->C|. Notice that we should see the value
\verb|a| as \verb|a : ()->A|. So we have the following more general fact: from
\verb|f : (A,B)->C| and \verb|a : A'->A| we can construct
\verb|(f a) :(A',B)->C|. Naturally, things should go on dually for coarities:
from \verb|f : A->(B,C)| and \verb|b : B->B'| one can build
\verb|(f b) : A->(B',C)|. We call a \emph{covalue} a value of type \verb|A*|
which is by definition \verb|A->()|. So we have in particular, from
\verb|f : A->(B,C)| and \verb|b : B*| one can build \verb|(f b) : A->C|.

So far so good. We can even mix both mechanisms: from \verb|f : A->(B,C)| and
\verb|g : (B,D)->E| one can build \verb|(g f) : (A,D)->(E,C)|. Now, suppose that
$\texttt{C}=\texttt{D}$. How do we ``finalize'' the composition? Using a trace!
But this is arguably not very natural...

Notice that in this case, there is really no notion of input and output since we
also have the following isomorphisms:
\begin{center}
  \verb|A->(B,C)|
  \qquad$\cong$\qquad
  \verb|(A,C*)->B|
  \qquad
  etc.
\end{center}

Since it might be good to work up to commutation, we could try a labeled
alternative, but this would be quite verbose because we would have to explicit
all the cuts. Moreover, it is not quite clear which instance of a function we
are using. For example, suppose that we want to compute the composite
\verb|h : (x:A)->(y:D)| of
\begin{center}
  \verb|f : (x:A)->(x':B,x'':C)|
  \qquad and\qquad
  \verb|g : (y':B,y'':C)->(y:D)|
\end{center}
This could be done as follows:
\begin{verbatim}
let f1 = new f
let g1 = new g
cut f1.x' g1.y'
cut f1.y'' g1.y''
let h = f1 * g1
\end{verbatim}
The details of this have to be thought in details and it is not very clear how
to do it in a way which is not too verbose. The idea of labels and colabels is
however interesting. In particular, optional coarguments mean that a default
continuation can be given.

\section{State of the art}
Wikipedia gives a good starting
point\footnote{\url{http://en.wikipedia.org/wiki/Comparison_of_audio_synthesis_environments}}.

% \begin{tabular}{cccc}
%   &\begin{sideways}blabla\end{sideways}\\
%   ChucK&\\
%   Csound&\\
%   Faust&\\
%   Impromptu&\\
%   Max&\\
%   Nyquist&
% \end{tabular}

\subsubsection{ChucK}
A textual programming language with a C-like syntax, compiled
on-the-fly\cite{chuck}. Their ``new operator'' called \emph{ChucK} and written
\texttt{=>} is a massively overloaded operator for various things such as
variable assignation, composition of operators, etc., which makes the language a
bit messy. Has a notion of \emph{event}. The language is globally messy (I
repeat).

\subsubsection{Csound}
Textual programming language with a syntax inherited from ancient
times~\cite{csound}.

\subsubsection{Faust}
\cite{faust}
Textual

\subsubsection{Impromptu}
\cite{impromptu} Textual with a Lisp syntax (Scheme). Very nice examples of ``live
coding'' of music by the author.

\subsubsection{Max}
\cite{max}
PureData is a rewrite of an old version of Max (by the same original author).

\subsubsection{Nsound}
A C++ framework~\cite{nsound}, ie just like ocaml-mm but in C++.

\subsubsection{Nyquist}
A textual programming language~\cite{nyquist}. It looks like a regular
programming language where all operators would implicitly take a time
parameter. Interesting notion of \emph{time warp} which locally modifies time by
another function.


\subsubsection{Pure Data}
A purely graphical graphical language~\cite{puredata}. Detailed
documentation~\cite{pd-doc}. The \emph{patches} (or \emph{canvases}) are build
from \emph{boxes} linked by wires. Two kind of data: \emph{messages} (our
events) and audio signals. There is a \emph{bang} message that boxes often
accept to trigger a computation. No detection of loops (stack overflows).

The language is not compiled, a toy compiler was
tried\footnote{\url{http://www.media.mit.edu/resenv/PuDAC/sw.html}} and
apparently already improves much performances. It would be nice to try with our compiler.

\subsubsection{PWGL}
\cite{pwgl}
Graphical

\subsubsection{Reaktor}
\cite{reaktor} Closed source. Tons of plugins. Graphical: nice GUI to control
synths + programming (\emph{patches} consist of \emph{modules} connected by
lines)

\subsubsection{SuperCollider}
Object-oriented textual programming language~\cite{supercollider}.

\section{Fun things to try when it's done}
\begin{itemize}
\item Lots of sounds generated in PD:
  \url{http://obiwannabe.co.uk/tutorials/html/tutorials_main.html}
  (and many nice other things on this site)
\item Lots of instruments and effects in Csound:
  \url{http://csounds.com/instruments}
\end{itemize}

\bibliographystyle{plain}
\bibliography{sound}
\end{document}
