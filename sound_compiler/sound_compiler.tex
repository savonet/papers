\documentclass{article}
\usepackage[utf8]{inputenc}
\usepackage[english]{babel}
\usepackage{hyperref}

\title{Ideas for\\An Optimizing Compiler\\for Sound Analysis and Synthesis}
\author{Samuel Mimram}

\hypersetup{
  pdftitle={\csname @title\endcsname},
  pdfauthor={\csname @author\endcsname},
  unicode=true,
  colorlinks=true,
  linkcolor=black,
  citecolor=black,
  urlcolor=black
}

\begin{document}
\maketitle

\section{What we want}
\subsection{What we should be able to do}
\begin{itemize}
\item We want to be able to manipulate streams of data: audio (streams of
  samples), video (streams of images), etc.
\item We should distinguish between two kind of things that we manipulate:
  \emph{values} which are always available in a stream (samples in audio data)
  and \emph{events} which are punctual data (metadata, notes, etc.)
\item It would be very nice to have a good typing of data represented in
  frequency domain (that is after an FFT): how do we integrate this properly in
  the language.
\end{itemize}

\subsection{What the language should look like}
\begin{itemize}
\item Operators with multiple outputs as well as multiple inputs. We really live
  in a monoidal category with biproducts.
\item What is a good syntax for this? We could have a categorical syntax (à la
  Faust) but having the opportunity to name wires seems to be nice, so both
  would be perfect (eg in $\lambda$-calculus we can do both $g\circ f$ and
  $\lambda x.g(f(x))$).
\item Typing system. We should begin with something simple.
\item We need a trace operator: instant feedback should be ruled out by either
  typing or checks for acyclicity.
\item How do we type clocks properly?
\item The language should be high-level and implementation-independent. In
  particular, it would be nice to be able to compile audio algos both to int and
  float and compare the difference of performance.
\end{itemize}

\subsection{Further optimizations}
\begin{itemize}
\item We should implement all the classical optimizations: constant propagation,
  expression factorization, etc.
\item Some optims are also specific to the domain of application: merge for
  loops as much as possible, etc.
\item Since the code will be very specific, we can vectorize it easily and use
  SIMD operations.
\item We should distinguish between three kind of values.
  \begin{enumerate}
  \item \emph{constant values}: never change during time
  \item \emph{sparse values}: rarely change during time. In practice, this means
    that we can consider them as constant during a buffer. The typical example
    is a value read from an user input such as a GUI.
  \item \emph{values} which are neither constant or sparse.
  \end{enumerate}
  The constant values might be inferred automatically. However, sparse value
  might need some annotations.
\item The stream data might be a simple value (floats for audio) or an allocated
  value (images). In the last case, we should avoid copies and we need to do
  this in a clever way.
\end{itemize}

\section{State of the art}
A good starting point:
\url{http://en.wikipedia.org/wiki/Comparison_of_audio_synthesis_environments}

\subsubsection{ChucK}

\subsubsection{Csound}
\cite{csound}
Textual

\subsubsection{Faust}
\cite{faust}
Textual

\subsubsection{Impromptu}
\cite{impromptu} Textual with a Lisp syntax (Scheme). Very nice examples of ``live
coding'' of music by the author.

\subsubsection{Max}
\cite{max}

\subsubsection{Nsound}
\cite{nsound} C++ framework

\subsubsection{Nyquist}
\cite{nyquist}
Textual

\subsubsection{Pure Data}
\cite{puredata}
Graphical

\subsubsection{PWGL}
\cite{pwgl}
Graphical

\subsubsection{Reaktor}
\cite{reaktor} Closed source. Tons of plugins. Graphical: nice GUI to control
synths + programming (\emph{patches} consist of \emph{modules} connected by
lines)

\subsubsection{SuperCollider}
\cite{supercollider}
Textual


\bibliographystyle{plain}
\bibliography{sound}
\end{document}
