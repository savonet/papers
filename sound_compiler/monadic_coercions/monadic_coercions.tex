\documentclass{article}
\usepackage[utf8]{inputenc}
\usepackage[english]{babel}
\usepackage{amsmath}
\usepackage{amssymb}
\usepackage{hyperref}
\usepackage{mathpartir}
\usepackage{amsthm}

\newtheorem{theorem}{Theorem}
\newtheorem{proposition}[theorem]{Proposition}
\theoremstyle{example}
\newtheorem{example}[theorem]{Example}

\renewcommand{\leq}{\leqslant}
\renewcommand{\geq}{\geqslant}
\renewcommand{\labelitemi}{--}
\newcommand{\cast}{\leq}
\newcommand{\qcast}{\quad\cast\quad}
\newcommand{\casted}[1]{\overline{#1}}
\newcommand{\qeq}{\quad=\quad}
\newcommand{\add}{\mathop{add}}
\newcommand{\FV}[1]{\mathop{FV}(#1)}
\newcommand{\TODO}[1]{\marginpar{\tiny #1}}
\newcommand{\gramdef}{\quad::=\quad}
\newcommand{\gramor}{\quad|\quad}
\newcommand{\trule}[1]{\mathrm{(#1)}}
\newcommand{\rewrites}{\rightsquigarrow}
\newcommand{\qqrewrites}{\qquad\rewrites\qquad}
\newcommand{\reduces}[1]{\longrightarrow_{#1}}
\newcommand{\mreduces}[1]{\overset\ast\longrightarrow_{#1}}

\title{Implicit Monadic Coercions}
\author{Samuel Mimram}
\hypersetup{
  pdftitle={\csname @title\endcsname},
  pdfauthor={\csname @author\endcsname},
  unicode=true,
  colorlinks=true,
  linkcolor=black,
  citecolor=black,
  urlcolor=black
}


\begin{document}
\maketitle

Monads are a classical notion in category theory~\cite{mac1998categories}. They
were introduced in computer science by Moggi~\cite{moggi2002computational} and
mad popular by Wadler~\cite{wadler1992comprehending}.

Mention Haskell's typeclasses and support for monads. However, this is not what
we want (explain why).

We want \emph{coercions} (or \emph{type conversion}).

What we want is sometimes called
\emph{autolifting}\footnote{\url{http://haskell.1045720.n5.nabble.com/Making-monadic-code-more-concise-td3266090.html}}.

An algorithm for coercing type inference in ML~\cite{luo2008coercions}: uses
return as a particular case of coercion, but the inferred type is
non-deterministic and thus not canonical.

A notion of subtyping between monad is studied in~\cite{tolmach1998optimizing}.

A completely unrelated paper with a cool title~\cite{shalgi2008being}.


\section{A $\lambda$-calculus with monadic types}
\subsection{The simply typed $\lambda$-calculus}
\paragraph{Terms.}
We suppose fixed a set of \emph{variables}~$x,y,\ldots$ and a set of
\emph{atomic terms} $t,u,\ldots$. $\lambda$-terms~$M$ are defined by the grammar
\[
M\gramdef x\gramor\lambda x.M\gramor MM\gramor t
\]
where~$x$ is a variable and~$t$ is an atomic term. Terms are considered up to
the usual $\alpha$-conversion rule.

The reduction is the usual $\beta$-reduction: $(\lambda x.M)N\reduces\beta
M[N/x]$, where $M[N/x]$ denotes the term~$M$ where every free occurrence of the
variable~$x$ has been replaced by the term~$N$. We also suppose that we are
given reduction rules of the form~$tM_1\ldots M_k\reduces\tau M$ where~$t$ is an
atomic term and reduction using these rules is called the $\tau$-reduction. We
say that a term~$M$ \emph{reduces} to~$N$ whenever~$M\mreduces{}N$,
where~$\mreduces{}$ is the reflexive and transitive closure
of~$\reduces{}=\reduces\beta\cup\reduces\tau$.

\paragraph{Types.}
We suppose fixed a set of \emph{atomic types}~$X,Y,\ldots$ and a set of
\emph{type constructors}~$T,U,\ldots$. We also suppose given a type~$A_t$ for
each atomic term~$t$. Types~$A$ are defined by the grammar
\[
A\gramdef X\gramor A\to A\gramor TA
\]
where~$X$ is an atomic type and~$T$ is a type constructor.

\paragraph{Typing rules.}
A \emph{context} $\Gamma$ is a set of pairs $\{x_1:A_1,\ldots,x_n:A_n\}$ where
the~$x_i$ are variables and the~$A_i$ are types. The variables~$x_i$ are called
the \emph{free variables} of~$\Gamma$ and denoted by~$\FV\Gamma$. A
\emph{sequent} is a triple $\Gamma\vdash M:A$ where~$\Gamma$ is a context, $M$
is a $\lambda$-term and~$A$ is a type.

Typing rules are defined by rules for deriving a sequent from a list of
sequents:
\[
\begin{array}{c@{\qquad\qquad}c}
  \inferrule{\Gamma\vdash M:A\to B\\\Gamma\vdash N:B}{\Gamma\vdash MN:B}{\trule{App}}
  &
  \inferrule{\Gamma,x:A\vdash M:B}{\Gamma\vdash\lambda x.M:A\to B}{\trule{Abs}}
  \\
  \inferrule{\null}{\Gamma,x:A\vdash x:A}{\trule{Ax}}
  &
  \inferrule{\null}{\Gamma\vdash t:A_t}{\trule{Cst}}
  \\
  \inferrule{\Gamma\vdash M:A}{\Gamma,x:B\vdash M:A}{\trule{Weak}}
\end{array}
\]
In the~$\trule{Abs}$ rule, we implicitly suppose that~$x$ is not free
in~$\Gamma$. In the following, we suppose that the $\tau$-reduction rules are
such that if $\Gamma\vdash M:A$ is derivable and $M\reduces\tau N$
then~$\Gamma\vdash N:A$ is also derivable, so that reduction always preserves
typing.

\subsection{Monads}
A monad $(T,r_T,b_T)$ consists of a type constructor~$T$ together with two
families of terms indexed by types~$A$
\[
r_T^A:A\to TA
\qquad\text{and}\qquad
b_T^A:(A\to TB)\to TA\to TB
\]
respectively called \emph{return} and \emph{bind} such that
\[
  b_Tk(r_T x)\mreduces{} kx
  \qquad
  b_Tr_Te\mreduces{} e
  \qquad
  b_T(\lambda x.b_Th(kx))e\mreduces{} b_Th(b_T k e)
\]
(we often omit the type superscripts for concision).

\begin{example}
  The list monad~$T$ is defined as follows. We write~$[x;y;z]$ for a list with
  three elements $x$, $y$ and~$z$. Given a type~$A$, $TA$ is the type for lists
  whose elements are of type~$A$. The return and bind function respectively have
  the following $\tau$-reduction rules:
  \[
  r_Tx\reduces\tau[x]
  \qquad\qquad
  b_Tfl\reduces\tau[x_1^1;\ldots;x_1^{k_1};x_2^1;\ldots;x_2^{k_i};\ldots\ldots;x_n^1;\ldots;x_n^{k_n}]
  \]
  where~$l=[x_1;\ldots;x_n]$ and~$f x_i=[x_i^1;\ldots;x_i^{k_i}]$. It can easily
  be checked that the axioms for monads are satisfied.
\end{example}

\subsection{Coercions}
Intuitively, the return operation~$r_T$ of a monad~$T$ allows us to
\emph{coerce} or \emph{cast} a value of type~$A$ into a value of type~$TA$, and
similarly the bind operation allows us to cast an argument of type~$TA$ of a
function into an argument of type~$A$. We would like to make these coercions
implicit. For instance, if $T$ is the list monad and the calculus contains usual
contants and types for maniuplating integers, then the term
\[
\add\ [1;2;3]\ 1
\]
should be automatically converted to
\[
(\lambda x'y.\ b_T\ (\lambda x.\ r_T\ (add\ x\ y))\ x')\ [1;2;3]\ 1
\]
which is typable and evaluates to~$[2;3;4]$.

Every monad thus defines a preorder relation on types, called \emph{casting}, as
the reflexive and transitive closure of the smallest relation~$\cast$ on types,
where \hbox{$A\cast B$} means that every value of type~$A$ may be casted into a
value of type~$B$, or more precisely that every typing proof on the left
of~\eqref{eq:cast} may be transformed in to a typing proof of the form given on
the right of~\eqref{eq:cast}, what we write
\begin{equation}
  \label{eq:cast}
  \inferrule{\pi}{\Gamma\vdash M:A}
  \qqrewrites
  \inferrule{\casted\pi}{\Gamma\vdash\casted{M}:B}
\end{equation}

The casting relation~$\cast$ is defined as the smallest relation on types such
that:
\begin{enumerate}
\item $A\cast TA$
\item $A\to TB\cast TA\to TB$
\item if~$B\cast B'$ then $A\to B\cast A\to B'$
\item if~$A\cast A'$ then $A'\to B\cast A\to B$
\item if $A\to C\cast A'\to C'$ then $A\to B\to C\cast A'\to B\to C'$
\item if $A\cast A'$ then $TA\cast TA'$
\end{enumerate}
The typing proof transformation is defined by the following rewriting rules:
\TODO{With bind we can also have the casting \hbox{$TA\cast (A\to TB)\to TB$} Do
  we want that?}
\begin{enumerate}
\item \emph{Return.}
  \[
  \inferrule{\pi}{\Gamma\vdash M:A}
  \qqrewrites
  \inferrule{
    \inferrule{\pi}{\Gamma\vdash M:A}
  }
  {\Gamma\vdash r_TM:TA}
  \]
\item \emph{Bind.}
  \[
  \inferrule{\pi}{\Gamma\vdash M:A\to TB}
  \qqrewrites
  \inferrule{
    \inferrule{\pi}{\Gamma\vdash M:A\to TB}
  }{\Gamma\vdash b_TM:TA\to TB}
  \]
\item \emph{Abstraction.}
  \[
  \inferrule{\pi}{\Gamma\vdash M:A\to B}
  \qqrewrites
  \inferrule{
    \inferrule{\casted\rho}{\Gamma,x:A\vdash \casted{Mx}:B'}
  }
  {\Gamma\vdash(\lambda x.\casted{Mx}):A\to B'}
  \]
  whenever
  \[
  \inferrule{\rho}{\Gamma,x:A\vdash Mx:B}
  \qqrewrites
  \inferrule{\casted\rho}{\Gamma,x:A\vdash\casted{Mx}:B'}
  \]
  where $\rho$ is the derivation
  \[
  \inferrule{
    \inferrule{
      \inferrule{\pi}{\Gamma\vdash M:A\to B}
    }
    {\Gamma,x:A\vdash M:A\to B}
  }
  {\Gamma,x:A\vdash Mx:B}
  \]
  and~$x\not\in\FV\Gamma$.
\item \emph{Argument restriction.}
  \[
  \inferrule{\pi}{\Gamma\vdash M:A'\to B}
  \qqrewrites
  \inferrule{
    \inferrule{
      \inferrule{
        \inferrule{\pi}{\Gamma\vdash M:A'\to B}
      }
      {\Gamma,x:A\vdash M:A'\to B}
      \\
      \inferrule{\casted\rho}{\Gamma,x:A\vdash\casted{x}:A'}
    }
    {\Gamma,x:A\vdash M\casted{x}:B}
  }
  {\Gamma\vdash\lambda x.M\casted{x}:A\to B}
  \]
  whenever
  \[
  \inferrule{\null}{\Gamma,x:A\vdash x:A}
  \qqrewrites
  \inferrule{\casted{\rho}}{\Gamma,x:A\vdash\casted{x}:A'}
  \]
\item \emph{Argument lifting.}
  \[
  \hspace{-10ex}
  \inferrule{\pi}{\Gamma\vdash M:A\to B\to C}
  \qqrewrites
  \inferrule{
    \inferrule{
      \inferrule{\casted\rho}{\Gamma,x:A',y:B\vdash\casted{\lambda x.Mxy}:A'\to C'}
      \\
      \inferrule{\null}{\Gamma,x:A',y:B\vdash x':A'}
    }
    {\Gamma,x:A',y:B\vdash\casted{(\lambda x.Mxy)}x':C'}
  }{\Gamma\vdash\lambda x'y.\casted{(\lambda x.Mxy)}x':A'\to B\to C'}
  \]
  whenever
  \[
  \inferrule{\rho}{\Gamma,x:A',y:B\vdash\lambda x.Mxy:A\to C}
  \qqrewrites
  \inferrule{\casted\rho}{\Gamma,x:A',y:B\vdash\casted{\lambda x.Mxy}:A'\to C'}
  \]
  where $\rho$ is the derivation
  \[
  \hspace{-15ex}
  \inferrule{
    \inferrule{
      \inferrule{
        \inferrule{
          \inferrule{\pi}{\Gamma\vdash M:A\to B\to C}
        }
        {\Gamma,x:A',y:B,x:A\vdash M:A\to B\to C}
        \and
        \inferrule{\null}{\Gamma,x:A',y:B,x:A\vdash x:A}
      }{\Gamma,x:A',y:B,x:A\vdash Mx:B\to C}
      \and
      \inferrule{\null}{\Gamma,x:A',y:B,x:A\vdash y:B}
    }
    {\Gamma,x:A',y:B,x:A\vdash Mxy:A\to C}
  }
  {\Gamma,x:A',y:B\vdash\lambda x.Mxy:A\to C}
  \]
  and~$x,y\not\in\FV\Gamma$.
\item \emph{Monad restriction.}
  \[
  \hspace{-13ex}
  \inferrule{\pi}{\Gamma\vdash M:TA}
  \qqrewrites
  \inferrule{
    \inferrule{
      \inferrule{\null}{\Gamma\vdash b_T:(A\to TA')\to TA\to TA'}
      \\
      \inferrule{\casted\rho}{\Gamma\vdash\casted{\lambda x.x}:A\to TA'}
    }
    {\Gamma\vdash b_T\casted{(\lambda x.x)}:TA\to TA'}
    \and
    \inferrule{\null}{\Gamma\vdash M:TA}
  }
  {\Gamma\vdash b_T\casted{(\lambda x.x)}M:TA'}
  \]
  whenever
  \[
  \inferrule{\rho}{\Gamma\vdash\lambda x.x:A\to A}
  \qqrewrites
  \inferrule{\casted\rho}{\Gamma\vdash\casted{\lambda x.x}:A\to TA'}
  \]
  where~$\rho$ is the derivation
  \[
  \inferrule{
    \inferrule{\null}{\Gamma,x:A\vdash x:A}
  }{\Gamma\vdash\lambda x.x:A\to A}
  \]
  and~$x\not\in\FV\Gamma$.
\end{enumerate}

\begin{proposition}
  For any rewriting
  \[
  \inferrule{\pi}{\Gamma\vdash M:A}
  \qqrewrites
  \inferrule{\casted{\pi}}{\Gamma\vdash\casted{M}:A'}
  \]
  of a derivation, we have that~$A\cast A'$.
\end{proposition}
\begin{proof}
  Immediate induction.
\end{proof}

\section{Properties of coercions}
\subsection{Determinism}
We suppose fixed monads~$(T_i,r_{T_i},b_{T_i})$ and write~$\equiv$ for the
smallest congruence containing $\alpha$-equivalence, $\beta$-reduction,
$\tau$-reduction, (typed) $\eta$-equivalence. Notice that the typing proof
rewriting rule~$\rewrites$ that we have defined is non-deterministic. However,
we show here that it is deterministic up to~$\equiv$.

\begin{theorem}
  Given types~$A$ and~$A'$ such that~$A\cast A'$, and a derivation
  \[
  \inferrule{\pi}{\Gamma\vdash M:A}
  \]
  if
  \[
  \inferrule{\pi}{\Gamma\vdash M:A}
  \rewrites
  \inferrule{\pi'}{\Gamma\vdash M':A'}
  \qquad\text{and}\qquad
  \inferrule{\pi}{\Gamma\vdash M:A}
  \rewrites
  \inferrule{\pi''}{\Gamma\vdash M'':A'}
  \]
  then~$M'\equiv M''$.
\end{theorem}

\subsection{Minimality}
Notice that~$\cast$ is not a partial order but only a preorder because for any
types~$A$ and~$B$, we have
\[
A\to TB
\qcast
TA\to TB
\qcast
A\to TB
\]
In the following we restrict.... TODO

A type~$A$ is \emph{expanded} when it is generated by the following grammar
\[
A\gramdef X\gramor A\to A\gramor TX
\]
where~$X$ is an atomic type. In expanded types monads are only allowed on atomic
types. For instance, given types~$A$ and~$B$, the type~$T(A\to B)$ is not
expanded.

In a given context~$\Gamma$, if~$M$ and~$N$ are terms of respective types~$TA$
and \hbox{$A\to B\to C$}, the term~$MN$ can be casted into a term of type
either~$TB\to TC$ or~$T(B\to C)$, which are incomparable wrt~$\cast$. In the
following, we restrict ourselves to expanded types.

\section{A typing algorithm}
\subsection{Inference up to coercion}
TODO: actually we should do everything preceding in this style because it
enables us to talk about applications which might not be well-typed!!!!

In the same spirit as~\cite{luo2008coercions}, we define inference of a term up
to coercion as derivations for sequents of the form
\[
\Gamma\vdash M\rewrites N:A
\]
where~$\Gamma$ is a context, $M$ and~$M'$ are terms and~$A$ is a type, meaning
that the term~$M$ coerces to the term~$M'$ which has type~$A$. Derivation rules are
\[
\begin{array}{c@{\qquad}c}
  \inferrule{\Gamma,x:A\vdash M\rewrites M':B}{\Gamma\vdash\lambda x.M\rewrites \lambda x.M':A\to B}{\trule{Abs}}
  &
  \inferrule{
    \Gamma\vdash M\rewrites M':A\to B
    \\
    \Gamma\vdash N\rewrites N':A
  }
  {\Gamma\vdash MN\rewrites M'N':B}{\trule{App}}
  \\[4ex]
  \inferrule{\Gamma\vdash M\rewrites M':A}{\Gamma\vdash M\rewrites r_T^AM':TA}{\trule{Ret}}
  &
  \inferrule{\Gamma\vdash M\rewrites M':A\to TB}{\Gamma\vdash M\rewrites b_T^AM':TA\to TB}{\trule{Bnd}}
  \\[4ex]
  \inferrule{\null}{\Gamma,x:A\vdash x\rewrites x:A}{\trule{Ax}}
  &
  \inferrule{\null}{\Gamma\vdash t\rewrites t:A_t}{\trule{Ax}}
  \\[4ex]
  \inferrule{\Gamma\vdash M\rewrites M':A}{\Gamma,x:B\vdash M\rewrites M':A}{\trule{Weak}}
  &
  \inferrule{\Gamma\vdash M\rewrites\lambda x.M'x:A\to B}{\Gamma\vdash M\rewrites M':A\to B}{\trule{Eta}}
  \\[4ex]
  \inferrule{\Gamma\vdash M\rewrites \lambda xy.(\lambda x M'xy)x:A\to B\to C}{\Gamma\vdash M\rewrites M':A\to B\to C}{\trule{Arg}}
  &
  TODO: \text{monad restriction}
\end{array}
\]

\begin{proposition}
  The proposed system is an extension of the usual typing system, in the
  following sense.
  \begin{enumerate}
  \item If $\Gamma\vdash M\rewrites M':A$ is derivable then~$\Gamma\vdash M':A$
    is derivable.
  \item If $\Gamma\vdash M:A$ is derivable then~$\Gamma\vdash M\rewrites M:A$ is
    derivable.
  \end{enumerate}
\end{proposition}

\begin{proposition}
  If $\Gamma\vdash M:A$ and~$\Gamma\vdash M\rewrites M':A'$ are derivable
  then~$A\cast A'$.
\end{proposition}

\subsection{The algorithm}

\subsection{Properties of the algorithm}
\begin{theorem}
  We infer the smallest possible type.
\end{theorem}

\section{An example}
Explain streams + feedback.

\section{Future works}
Extend to more general cases.
\begin{itemize}
\item Composition of monads: via a distributive law~$l_{TU}:TUA\to UTA$.
\item Remove the restriction to expanded type. This can be done if we suppose
  that monads we consider are equipped with a strength $s_T:T(A\to B)\to TA\to
  TB$. In this case, we can add the relation~$T(A\to B)\cast TA\to TB$ and adapt
  the algorithms.
\item Other constructions such as products and coproducts. Similar to the
  previous cases, we should add strengths such as~$p_T:T(A\times B)\to TA\times
  TB$.
\item Type schemes with Hindley-Milner type
  inference~\cite{hindley1969principal, damas1982principal}. Notice that it
  breaks let-polymorphism: each instance of a value declared by a let has to be
  compiled separately. Also, it is ambiguous with polymorphism: $id\ [1;2;3]$
  should evaluate (in one step) to~$[1;2;3]$ or to~$[id\ 1;id\ 2;id\ 3]$? We
  should be able to obtain results with parametricity though... (in the previous
  example, both evaluations lead to the same result!)
  see~\cite{wadler1989theorems}.
\end{itemize}

\bibliographystyle{alpha}
\bibliography{monadic}
\end{document}
