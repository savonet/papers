\documentclass{article}

\usepackage{amsmath}
\usepackage{proof}
\renewcommand{\|}{\; | \;}

\begin{document}

The language at this stage is \textbf{not a stream language}.
It is only a language about functions with multiple outputs
or, from a logical viewpoint, functions of type
$A_1 \wedge \ldots \wedge A_n \supset B_1 \wedge \ldots \wedge B_n$.

The main reason for designing such a language is to give more
flexibility than with tuples, and notably compose multiple ``wires''
at one time.

Here is an alternative approach to build such a language:
we design a sequent calculus with the expected operations,
and could derive a term language for that.

Our sequents represent a conjunction of hypothesis entailing
a \textbf{conjunction} of conclusions. We only put the structural
rules that have a (canonical) meaning for us, and it's all pretty
straightforward.

\section{Explicit presentation}

In the following I write sequents treating both sides as multisets,
but really the formulas come with a location/name distinguishing them.

Identity:
\[ \infer{A\vdash A}{} \]

Contraction:
\[ \infer{\Gamma,A\vdash \Delta}{\Gamma,A,A\vdash \Delta}
\quad\quad
   \infer{\Gamma\vdash A,A,\Delta}{\Gamma\vdash A,\Delta} \]

Weakening:
\[ \infer{\Gamma,A\vdash\Delta}{\Gamma\vdash\Delta}
\quad\quad
   \infer{\Gamma\vdash \Delta}{\Gamma\vdash A,\Delta} \]

Logical rules:
\[ \infer{(\Gamma)\vdash\top}{} \quad\quad \infer{\Gamma,\bot\vdash\Delta}{} \]
\[ \infer{\Gamma,A\wedge A'\vdash \Delta}{\Gamma,A,A'\vdash \Delta}
\quad\quad
   \infer{\Gamma\vdash A\wedge A',\Delta}{\Gamma\vdash A,A',\Delta} \]
\[ \infer{\Gamma,A\vee A' \vdash \Delta}{\Gamma,A\vdash \Delta &
                                         \Gamma,A'\vdash \Delta}
\quad\quad
   \infer{\Gamma \vdash A_0\vee A_1,\Delta}{\Gamma\vdash A_i,\Delta} \]
\[ \infer{\Gamma,\Gamma',B\supset A\vdash \Delta,\Delta'}{
      \Gamma\vdash B,\Delta & \Gamma',A\vdash \Delta'}
\quad\quad
   \infer{\Gamma\vdash A\supset B}{\Gamma,A\vdash B} \]
\[ \infer{\Gamma,\Gamma'\vdash \Delta,\Delta'}{
      \Gamma\vdash\Delta & \Gamma'\vdash\Delta'} \]

The right implication rule may seem restrictive but it is important to avoid
proving $(A\supset B)\wedge \Gamma$ by using $A$ to produce $\Gamma$:
for example, $(A\supset A),A$ should not be provable.
The left rule seems to make sense; still it's worth checking cut-elimination
carefully (all other rules are harmless).

A sequent with an empty right hand-side should be provable,
so we could reformulate the $\top$ rule to simply erase $\top$.

The last ``splitting'' rule is necessary because of the two
rules that force having only one conclusion: the
right implication rule and the identity.
It can be used to build a proof of $(A \supset A),\top$.

\section{Optimized presentation}

As usual we can get rid of contraction and weakening on the left.

Identity (the new variant avoids using the splitting rule,
but in a real system we would probably need the two forms):
\[ \infer{\Gamma,A\vdash A,\Delta}{\Gamma,A\vdash \Delta} \]

Contraction:
\[ \infer{\Gamma\vdash A,A,\Delta}{\Gamma\vdash A,\Delta} \]

Weakening:
\[ \infer{\Gamma\vdash \Delta}{\Gamma\vdash A,\Delta} \]

Logical rules:
\[ \infer{\Gamma\vdash\top,\Delta}{\Gamma\vdash\Delta}
    \quad\quad \infer{\Gamma\vdash}{}   
    \quad\quad \infer{\Gamma,\bot\vdash\Delta}{} \]
\[ \infer{\Gamma,A\wedge A'\vdash \Delta}{\Gamma,A,A'\vdash \Delta}
\quad\quad
   \infer{\Gamma\vdash A\wedge A',\Delta}{\Gamma\vdash A,A',\Delta} \]
\[ \infer{\Gamma,A\vee A' \vdash \Delta}{\Gamma,A\vdash \Delta &
                                         \Gamma,A'\vdash \Delta}
\quad\quad
   \infer{\Gamma \vdash A_0\vee A_1,\Delta}{\Gamma\vdash A_i,\Delta} \]
\[ \infer{\Gamma,B\supset A\vdash \Delta,\Delta'}{
      \Gamma\vdash B,\Delta & \Gamma,A\vdash \Delta'}
\quad\quad
   \infer{\Gamma\vdash A\supset B}{\Gamma,A\vdash B} \]
\[ \infer{\Gamma\vdash \Delta,\Delta'}{
      \Gamma\vdash\Delta & \Gamma\vdash\Delta'} \]

I finally add the $n$-ary cut:
\[ \infer{\Gamma\vdash \Delta,\Delta'}{
       \Gamma\vdash \Delta,\Delta'' & \Delta'',\Gamma\vdash\Delta'} \]

\section{Term language}

Now I derive a stupid term language from the above rules,
or rather their natural deduction equivalent,
using locations ($l$) to disambiguate:
\begin{eqnarray*}
 M &::=& \text{insert}_l(x,M) \| \text{dup}_l(M) \| \text{del}_l(M) \\
   &\|& \text{insert}_l((),M) \| () \| \delta_{\bot,l}(M) \\
   &\|& \text{let } x,y = M \text{ in } M' \\
   &\|& \text{pair}_{l_1,l_2\mapsto l}(M) \\
   &\|& \delta_\vee(M,M',M'') \| \text{in}_i(M) \\
   &\|& M_1 M_2 \| \lambda x. M \\
   &\|& (M_1,M_2) \\
   &\|& \text{cut}_{l_1,\ldots,l_n\mapsto l'_1,\ldots,l'_n}(M_1,M_2)
\end{eqnarray*}

It looks like insertion is used twice but notice that it is not a general
insertion construct, but can only be used to insert a variable or unit
(ie. an axiom or top proof).
But in a practical language, the general operation might make sense.

We probably also want a construct for swapping wires,
it should be derivable.

For readability,
the natural deduction reformulation of application (and disjunction 
elimination to a lesser extent) is simplified with arities one,
but the general form should
be derivable (our calculus is far from being minimal, in fact it's just
intuitionistic logic with lots of noise).

Note that cut and application are not the same: application applies on one
wire, but cut really composes proofs by connecting several wires.

\end{document}
