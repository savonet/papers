
% DB les sous-sections peuvent virer, surtout pour faire de la place
%   par contre l'organisation globale doit rester: on raconte ce qu'il
%   se passe au niveau des sources, pour bien montrer que ce qu'on fait
%   dans le typage correspond précisemment aux contraintes imposées
%   par le modele
% \subsection{Model}

% Meme si ya le mot sous-typage, cette relation est vraiment utilisée
% dans le modele indépendamment du langage, d'ailleur c'est
% Frame.kind_sub_kind. Après on copie tout ça au niveau du langage.

In \liquidsoap{}, streams can contain data of various nature. The typical
example is the case of video streams which usually contain both images and
audio samples. We also support MIDI streams (which contain musical notes)
and it would be easy to add other kinds of content.
% which allows us to provide
% ``synthesizer'' operators
% (which take a stream of notes as input and produce an
% audio stream).
% In addition, there can be multiple channels of data for each data
% kind: audio is usually stored in stereo (2~channels), but also sometimes in
% surround sound (5~channels), etc.
It is desirable to allow sources of different
content kinds within a streaming system, which makes it necessary to
introduce a typing discipline in order to ensure 
the consistency of stream contents across sources.
% (\eg some sources might produce sound and other only video and they are merged
% afterward)

The nature of data in streams is described by its \emph{content type}, which is
a triple of natural numbers indicating the number of audio, video and midi
channels.  A stream may not always contain data of the same type.  For instance,
the \texttt{playlist} operator might rely on decoding files of heterogeneous
content, \eg\ mono and stereo audio files.  In order to specify how content
types are allowed to change over time in a stream, we use \emph{arities}, which
are essentially natural numbers extended with a special symbol $\star$:
\[
a\quad ::=\quad \star \;|\; 0 \;|\; S(a)
\]
An arity is \emph{variable} if it contains $\star$, otherwise it is an usual
natural number, and is \emph{fixed}. A \emph{content kind} is a triple of
arities, and specifies which content types are acceptable. For example,
$(S(S(0)),S(\star),\star)$ is the content kind meaning ``2 audio channels, at
least one video channel and any number of MIDI channels''.
% S<:T means "if you can work with T you can work with S"
%            "if you can work with a source(T) you can take a source(S)"
%            "T is more permissive than S"
This is formalized through the subtyping relation defined in
Figure~\ref{fig:subtyping}: $T\sub K$ means
that the content kind $T$ is allowed by $K$. More generally,
$K \sub K'$ expresses that $K$ is more permissive than $K'$,
which implies that a source of content kind $K$ can safely be seen
as one of content kind $K'$.

\begin{figure}[htpb]\[
% c'est possible de simplifier en mettant direct K<:K,
% mais on ne peut pas dire K<:* sans vérifier que K est bien formé
   \infer{0\sub 0}{} \quad\quad
   \infer{S(A)\sub S(A')}{A\sub A'} \quad\quad
   \infer{\star\sub\star}{} \quad\quad
   \infer{0\sub \star}{} \quad\quad
   \infer{S(A)\sub \star}{A\sub\star}
\]\[
   \infer{(A,B,C) \sub (A',B',C')}{A \sub A' & B\sub B' & C\sub C'}
\]
 \fcaption{Subtyping relation on arities}
 \label{fig:subtyping}
\end{figure}

When created, sources are given their expected content kind.
Of course, some assignments are invalid.
For example,
a pure audio source cannot accept a content kind which requires video 
channels, and many operators cannot produce a stream of an other kind
than that of their input source.
Also, some sources have to operate on input streams that have a fixed kind --
a kind is said to be fixed when all of its components are.
This is the case of the \texttt{echo} operator which produces echo on sound
and has a internal buffer of a fixed format for storing past sound,
or sound card inputs/outputs which have to initialize the sound card for
a specific number of channels.
Also note that passing the expected content kind
is important because some sources behave differently depending on their kind,
as shown with the previous example.

\label{sec:typing-ex}
\begin{figure}[b]
  \centering
  \texttt{
    \begin{tabular}{rcl}
    swap&:&(source(2,0,0)) -> source(2,0,0)\\
    on\_metadata&:&(handler,source('*a,'*b,'*c)) -> source('*a,'*b,'*c)\\
    % id&(source('a,'b,'c)) -> source('a,'b,'c)\\
    echo&:&(delay:float,source('\#a,0,0)) -> source('\#a,0,0)\\
    % DB vire le prefixe "video." pour faire de la place a gauche
    %  en vrai greyscale veut du fixed arity, pcq son implem est pas
    %  assez générale
    greyscale&:&(source('*a,'*b+1,'*c)) -> source('*a,'*b+1,'*c)\\
    output.file &:& 
       (format('*a,'*b,'*c),string,source('*a,'*b,'*c))->\\
     & & source('*a,'*b,'*c) \\
     & &
    \end{tabular}
  }
  \fcaption{Types for some operators}
  \label{fig:types}
\end{figure}

\paragraph{Integration in the language.}
To ensure that streaming systems built from user scripts will never
encounter situations where a source receives data that it cannot handle,
we leverage various features of our type system.
By doing so, we guarantee statically that content type mismatches never happen.
The content kinds are reflected into types,
and used as parameters of the \texttt{source} type.
In order to express the types of our various operators,
we use a couple features of type systems
(see \cite{pierce02book} for extensive details).
As expected, the above subtyping relation is integrated into
the subtyping on arbitrary \liquidsoap\ types.
We illustrate various content kinds in the examples of Figure~\ref{fig:types}:
\begin{itemize}
\item The operator \texttt{swap} exchanges the two channels of a stereo audio
  stream. Its type is quite straightforward: it operates on streams with exactly
  two audio channels.
\item
  \liquidsoap\ supports polymorphism \emph{à la} ML.
  We use it in combination with constraints to allow arbitrary arities.
  The notation \verb.'*a. stands for a universal variable (denoted
  by \verb.'a.) to which a type constraint is attached, expressing that
  it should only be instantiated with arities.
  For example, the operator \verb.on_metadata. does not rely
  at all on the content of the stream, since it is simply in
  charge of calling a handler on each of its metadata packets --
  in the figure, \verb.handler. is a shortcut for
  \verb.([string*string]) -> unit..
\item When an operator, such as \verb.echo., requires a fixed content type, we
  use another type constraint. The resulting constrained universal
  variable is denoted by \verb.'#a. and can only be instantiated with
  fixed arities.
\item The case of the \texttt{greyscale} operator, which converts a color
  video into greyscale, shows how we can require at least one video channel in
  types.  Here, \verb.'*b+1. is simply a notation for \verb.S('*b)..
\item Finally, the case of \verb#output.file# (as well as several other outputs
  which encode their data before sending it to various media) is quite
  interesting. Here, the expected content kind depends on the format the stream
  is being encoded to, which is given as first argument of the operator. Since
  typing the functions generating formats would require dependent types (the
  number of channels would be given as argument) and break type inference, we
  have introduced particular constants for type formats with syntactic sugar for
  them to appear like functions -- similar ideas are for example used to type
  the \texttt{printf} function in OCaml. For example,
  \verb$output.file(%vorbis,"stereo.ogg",s)$ requires that
  \verb.s. has type \verb.source(2,0,0). because \verb.%vorbis. alone
  has type \verb.format(2,0,0).,
  but \verb$output.file(%vorbis(channels=1),"mono.ogg",s)$
  requires that there is only one audio channel;
  we also have video formats such as \verb.%theora..
\end{itemize}

These advanced features of the type system are statically inferred, which means
that the gain in safety does not add any burden on users. As said above, content
kinds have an influence on the behavior of sources |
polymorphism is said to be \emph{non-parametric}.
In practice, this means that static types must be
maintained throughout the execution of a script. This rather unusual aspect
serves us as an overloading mechanism: the only way to remove content kinds from
execution would be to duplicate our current collection of operators with a
different one for each possible type instantiation.
% \TODO{SM c'est pas très clair cette dernière phrase \\
%   DB ouais, c'est techniquement dense, j'arrive pas à faire mieux
%    mais je vais laisser pour contrer les puristes du parametric polymorphism}

%% SM c'est du détail et on a pas bcp de place
% Type inference leaves unknown types, they are forced using the default
% number of channels, and users can write type annotations.
% But in most cases, type inference works pretty well, since most
% outputs dictate content kinds.
